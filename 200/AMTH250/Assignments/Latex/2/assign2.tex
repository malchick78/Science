\documentclass[11pt,a4paper]{article}

\usepackage{amsmath}

\title{AMTH250 \hfill Assignment 2}

\author{Mark Villar}

\begin{document}

\maketitle

\subsubsection*{Question 1}
The reduced cubic equation $y^{3}+3py+2q=0$ has one real and two complex solutions when $D=q^{2}+p^{3}>0$. These are given by Cardano's formula as 
	\begin{align*}
		y_{1}=u+v, \quad y_{2}=-\frac{u+v}{2}+\frac{i}{2}\sqrt{3}(u-v), \quad y_{3}=-\frac{u+v}{2}-\frac{i}{2}\sqrt{3}(u-v)
	\end{align*}
	where
	\begin{align*}
		u=\sqrt[3]{-q+\sqrt{q^{2}+p^{3}}},  \qquad v=\sqrt[3]{-q-\sqrt{q^{2}+p^{3}}}
	\end{align*}

\subsubsection*{Question 2}
An $n \times n \ \text{matrix} \ \mathbf{A}$ is \textsl{non-singular} if it satisfies any one of the following equivalent conditions:
	\begin{enumerate}
		\item $\mathbf{A}$ has an inverse, i.e. there is a matrix $\mathbf{A}^{-1}$ such that 
		\begin{align*}
			\mathbf{AA}^{-1}=\mathbf{A}^{-1}\mathbf{A}=\mathbf{I}
		\end{align*}
		\item $\det \mathbf{A} \neq 0.$

		\item $\operatorname{rank} \mathbf{A} = n$. (The rank of a matrix is the maximum number of linearly independent rows or columns of the matrix.)

		\item For every non-zero vector $\mathbf{z}$, $\mathbf{Az} \neq 0$.

	\end{enumerate}

\subsubsection*{Question 3}
The \textit{gamma function} $\Gamma(x)$ is defined by 
	\begin{align*}
		\Gamma(x) &\equiv \lim_{n \rightarrow \infty} \prod^{n-1}_{\nu=0} \frac{n! \thinspace n^{x-1}}{x+\nu} \\
		&=\lim_{n \rightarrow \infty} \frac{n! \thinspace n^{x-1}}{x(x+1) \cdots (x+n-1)} \\
		&=\int^\infty_0 e^{-t}t^{x-1} \thinspace dt
	\end{align*} 
The integral definition is only valid for $x>0$.

\subsubsection*{Question 4}
Given an $n$-vector $\mathbf{a}$, we can annihilate \textbf{all} of its entries below the $k$th position, provided that $a_k \neq 0$, by the following transformation:
	\begin{align*}
		\mathbf{M}_k\mathbf{a}=
		\begin{bmatrix}
			1 &\ldots &0 &0 &\ldots &0 \\
			\vdots &\ddots &\vdots &\vdots &\ddots &\vdots \\
			0 &\ldots &1 &0 &\ldots &0 \\
			0 &\ldots &-m_{k+1} &1 &\ldots &0 \\
			\vdots &\ddots &\vdots &\vdots &\ddots &\vdots \\
			0 &\ldots &-m_n &0 &\ldots &1
		\end{bmatrix}
		\begin{bmatrix}
			a_1 \\
			\vdots \\
			a_k \\
			a_{k+1} \\
			\vdots \\
			a_n
		\end{bmatrix}=
		\begin{bmatrix}
			a_1 \\
			\vdots \\
			a_k \\
			0 \\
			\vdots \\
			0
		\end{bmatrix}
	\end{align*}
where $m_i=a_i/a_k, \thinspace i=k+1, \ldots, n$. 

\subsubsection*{Question 5}
The absolute and relative error are defined by
	\begin{align}
		\text{Absolute error } \thickspace &= \thickspace \ \text{Approximate value} - \text{True value} \\
		\text{Relative error } \thickspace &= \thickspace \ \frac{\text{Absolute error}}{\text{ True value}}
		\intertext{A useful way to think of relative error is via the expression}
		\text{Approximate value } \thickspace &= \thickspace \ \text{True value} \times (1+\text{Relative error})
	\end{align}
	
\subsubsection*{Question 6}
Differentiating the differential equation 
	$$\frac{dy}{dt}=f(t,y)$$
gives
	$$\frac{d^2y}{dt^2}=\frac{d}{dt}f(t,y)=\frac{\partial f}{\partial t}+\frac{\partial f}{\partial y}\frac{dy}{dt}=\frac{\partial f}{\partial t}+\frac{\partial f}{\partial y}f$$
and we have the Taylor series approximation
	\begin{align*}
		y(t+h) &\approx y(t)+y'(t)h+\frac{1}{2}y''(t)h^2 \\
		&=y(t)+hf(t,y)+\frac{h^2}{2}\left(\frac{\partial f}{\partial t}(t,y)+\frac{\partial f}{\partial y}(t,y)f(t,y)\right)
	\end{align*}

\newpage
	
\subsubsection*{Question 7}
	\begin{table}[!ht]
		\begin{center}
			\begin{tabular}{|c|c|c|c|c|c|}
				\hline
				\multicolumn{2}{|c|}{No Pivoting} &\multicolumn{2}{c|}{Partial Pivoting} &\multicolumn{2}{c|}{Complete Pivoting} \\
				\hline
				Error &Residual &Error &Residual &Error &Residual \\
				\hline
				8.964 &$2.07 \times 10^{-14}$ &0.156 &$2.53 \times 10^{-16}$ &0.164 &$2.53 \times 10^{-16}$ \\
				1.426 &$2.77 \times 10^{-15}$ &0.113 &$2.04 \times 10^{-16}$ &0.175 &$2.93 \times 10^{-16}$ \\
				0.883 &$3.60 \times 10^{-16}$ &0.080 &$2.97 \times 10^{-16}$ &0.036 &$3.48 \times 10^{-16}$ \\
				\hline
			\end{tabular}
		\end{center}
		\caption{Errors and residuals for 3 random $100 \times 100$ matrices.}
	\end{table}

\subsubsection*{Question 8}
	\begin{center}
		\begin{tabular}{cc} % main table
			\multicolumn{2}{c}{Payoff (\$)} \\
			\multicolumn{1}{c}{Player 1} &\multicolumn{1}{c}{Player 2} \\
			\hline
			\\
			\begin{tabular}{|c|c|c|} % sub-table 1
				\hline 
				1 &2 &3 \\
				\hline
				4 &5 &6 \\
				\hline
			\end{tabular} &
			\begin{tabular}{|c|c|} % sub-table 2
				\hline
				1 &3 \\
				\hline
				2 &5 \\
				\hline
				3 &6 \\
				\hline
			\end{tabular} \\
		\end{tabular}
	\end{center}
\end{document}