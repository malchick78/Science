\documentclass[12pt]{amsart}

\usepackage{a4wide, amsxtra}

\usepackage[pdftex]{graphicx}

 \title{PMTH212 Assignment 4}

 \author{Mark Villar}

\begin{document} 

\maketitle 

\begin{enumerate}
	
	\item 
	
		\begin{enumerate}
		
			\item $f(x,y)=(x-y)^{\frac{1}{2}}=\sqrt{x-y} \ \Rightarrow \ x-y>0 \equiv y<x$
									
			\item $f(x,y)=\cos\left(\dfrac{xy}{1+x^2+y^2}\right) \ \Rightarrow \ 1+x^2+y^2\ne0$ \\
					 Since $1+x^2+y^2 \ge1$ for all $x,y$ then $\mathbb{R}^2$ is the domain of $f$. We also conclude that $f$ is continuous over its entire domain since  
					 $$\lim_{(x,y)\rightarrow(x_0,y_0)} f(x,y)=f(x_0,y_0)$$
			
		\end{enumerate}
		
		\medskip
		
	\item
		
		\begin{enumerate}
		
			\item We simplify the function and observe continuity over its entire domain $\mathbb{R}^2$. Hence the limit is simply the value of the function at $(0,0)$.
				\begin{align*}
					\lim_{(x,y) \rightarrow (0,0)} \frac{x^4-16y^4}{x^2+4y^2} = \lim_{(x,y) \rightarrow (0,0)} x^2-4y^2 = 0-0 = 0
				\end{align*}
			
			%We also check the limit along the $x$-axis $(y=0)$:
				%\begin{align*}
					%\lim_{\text{along} \ x\text{-axis}} \frac{x^4-16y^4}{x^2+4y^2} = \lim_{x \rightarrow 0} \ x^2 = 0
				%\end{align*}
				
			%and the limit along the $y$-axis $(x=0)$:
				%\begin{align*}
					%\lim_{\text{along} \ y\text{-axis}} \frac{x^4-16y^4}{x^2+4y^2} = \lim_{y \rightarrow 0} \ -4y^2 = 0
				%\end{align*}
			
			\item Let $z=x^2+y^2$ \ and \ $z \rightarrow 0$ \ if \ $(x,y) \rightarrow (0,0)$.
				\begin{align*}
					\lim_{(x,y)\rightarrow(0,0)} \frac{\sin(x^2+y^2)}{x^2+y^2} = \lim_{z \rightarrow 0} \frac{\sin z}{z} = 1
				\end{align*}
		
		\end{enumerate}
		
		\medskip
					
	\item
		
		\begin{enumerate}
		
			\item $C_m: y=mx$ \ or \ $x=t, \  y=mt$. Let $(x,y) \rightarrow (0,0)$ \ along \ $C_m.$ \ Since \ $m \ne 0,$
				\begin{align*}
					\lim_{(x,y)\rightarrow(0,0)} \ \frac{x^3y}{2x^6+y^2} = \lim_{t\rightarrow 0} \ \frac{t^3(mt)}{2t^6+(mt)^2} = \lim_{t\rightarrow 0} \ \frac{mt^2}{2t^4+m^2}=0
				\end{align*}
				
				$C_k: y=kx^2$ \ or \ $x=t, \ y=kt^2$. Let $(x,y) \rightarrow (0,0)$ \ along \ $C_k.$ \ Since \ $k \ne 0,$
				\begin{align*}
					\lim_{(x,y)\rightarrow(0,0)} \ \frac{x^3y}{2x^6+y^2} = \lim_{t\rightarrow 0} \ \frac{t^3(kt^2)}{2t^6+(kt^2)^2} = \lim_{t\rightarrow 0} \ \frac{kt}{2t^2+k^2}=0
				\end{align*}
									
			\item $C_r: y=rx^3$ \ or \ $x=t, \ y=rt^3$. Let $(x,y) \rightarrow (0,0)$ \ along \ $C_r.$  \ Since \ $r \ne 0,$
				\begin{align*}
					\lim_{(x,y)\rightarrow(0,0)} \ \frac{x^3y}{2x^6+y^2} = \lim_{t\rightarrow 0} \ \frac{t^3(rt^3)}{2t^6+(rt^3)^2} = \lim_{t\rightarrow 0} \ \frac{r}{2+r^2}
				\end{align*}
				
				By choosing different values for $r$ we obtain different limits along $C_r$, implying that the limit does not exist. Moreover, since the limits along $C_m$ and $C_k$ are different to the limit along $C_r,$ we conclude that the function does not have a limit as $(x,y) \rightarrow (0,0)$.
			
		\end{enumerate}
		
	\item $$f(x,y,z)=z\ln(x^2y\cos z) = z(2\ln x+\ln y+\ln \cos z)$$
		$$f_x=\dfrac{2z}{x}, \ \ f_y=\dfrac{z}{y}, \ \ f_z=\ln \cos z-z\tan z$$ 
		Note:
			\begin{align*}
				w&=\ln u, \ u=\cos z \\
				\frac{dw}{dz}&=\frac{dw}{du}\cdot\frac{du}{dz}=\frac{1}{u}\cdot(-\sin z) \\
				&=-\frac{\sin z}{\cos z}=-\tan z
			\end{align*}
			
			\bigskip
			
	\item
	
		\begin{enumerate}
		
			\item $f(x,y)=\ln(x^2+y^2)$
				\begin{align*}
					&\frac{\partial f}{\partial x}=\frac{2x}{x^2+y^2} \ , \ \ \ \ \ \ \ \ \ \ \ \ \ \ \ \ \ \frac{\partial f}{\partial y}=\frac{2y}{x^2+y^2} \\
					&\frac{\partial^2 f}{\partial x^2}=-\frac{2(x^2+y^2)-2x(2x)}{(x^2+y^2)^2}=\frac{2(y^2-x^2)}{(x^2+y^2)^2} \\
					&\frac{\partial^2 f}{\partial y^2}=-\frac{2(x^2+y^2)-2y(2y)}{(x^2+y^2)^2}=\frac{2(x^2-y^2)}{(x^2+y^2)^2}
				\end{align*}
				\begin{align*}
					\ \ \ \ \frac{\partial^2 f}{\partial x^2}+\frac{\partial^2 f}{\partial y^2}=\frac{2(y^2-x^2)}{(x^2+y^2)^2}+\frac{2(x^2-y^2)}{(x^2+y^2)^2}=\frac{2(y^2-x^2)+2(x^2-y^2)}{(x^2+y^2)^2}=0
				\end{align*}
				
				\bigskip
									
			\item $u(x,y)=e^x\cos y, \ v(x,y)=e^x\sin y$ 
				\begin{align*}
					&\frac{\partial u}{\partial x}=e^x \cos y \ , \ &\frac{\partial v}{\partial y}=e^x \cos y \\
					&\frac{\partial u}{\partial y}=-e^x \sin y \ , \ &\frac{\partial v}{\partial x}=e^x \sin y
				\end{align*}
				
				\medskip 
				
			Hence, $\partial u/\partial x = \partial v/\partial y$ \ and \ $\partial u/\partial y = -\partial v/\partial x$
			
				\bigskip
			
		\end{enumerate}	
				
	\item $f(x,y)=(x^2+y^2)^{2/3}$
		\begin{align*}
			%&f_x(x,y)=\frac{2}{3}(x^2+y^2)^{-1/3}(2x)=\frac{4x}{3\sqrt[3]{x^2+y^2}}
			&f(x,0)=(x^2+0)^{2/3}=x^{4/3} \\
			&f_x(x,0)=\frac{4}{3}x^{1/3} \\
			&f_x(0,0)=\frac{4}{3}(0)=0
		\end{align*}
				
\end{enumerate}

\end{document}
