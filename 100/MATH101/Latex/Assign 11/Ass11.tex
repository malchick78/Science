\documentclass[12pt]{amsart}

\usepackage{a4wide, amsxtra}

\usepackage[pdftex]{graphicx}

\usepackage{hyperref}

 \title{MATH101 Assignment 11}

 \author{Mark Villar}

\begin{document} 

\maketitle 

\begin{enumerate}
	
	\item To show that the system is inconsistent if $\gamma\ne2\alpha-3\beta$, we can write the equivalent
		statement: the system is consistent if and only if $\gamma=2\alpha-3\beta$.  We now use 
		Gauss-Jordan elimination to solve for the three unknowns $x, y$ and $z$.
		\\
		Assume $\gamma=2\alpha-3\beta$. The augmented matrix then becomes
		\begin{align}
			\begin{array}{ccc|c}
				\hspace{1.2cm} \text{ } \text{ } 2&-1& \text{ } \text{ } 3& \alpha \\
				\hspace{1.2cm} \text{ } \text{ } 3& \text{ } \text{ } 1&-5& \beta  \\
				\hspace{1.2cm} -5& -5& \text{ } \text{ }21& 2\alpha-3\beta
			\end{array} \notag
		\end{align} 
		Next we convert the matrix to its reduced row echelon form.
		\begin{align}
			\begin{array}{ccc|c}
				R_1+R_2	 \hspace{0.55cm} \text{ } 5&\text{ } \text{ } 0&-2& \alpha+\beta\\
				\hspace{1.9cm} \text{ } \text{ } 3& \text{ } \text{ } 1&-5& \beta  \\
				\hspace{1.85cm} -5& -5& \text{ } \text{ }21& 2\alpha-3\beta
			\end{array} \notag
		\end{align} 
		\begin{align}
			\begin{array}{ccc|c}
				\hspace{1.95cm} \text{ } \text{ } 5&\text{ } \text{ } 0&-2& \alpha+\beta\\
				R_2+R_1	\hspace{0.5cm} \text{ } \text{ } 8& \text{ } \text{ } 1&-7& \alpha+2\beta  \\
				\hspace{1.85cm} -5& -5& \text{ } \text{ }21& 2\alpha-3\beta
			\end{array} \notag
		\end{align} 
		\begin{align}
			\begin{array}{ccc|c}
				\hspace{1.95cm} \text{ } \text{ } 5&\text{ } \text{ } 0&-2& \alpha+\beta\\
				\hspace{1.95cm} \text{ } \text{ } 8& \text{ } \text{ } 1&-7& \alpha+2\beta  \\
				R_3+R_1 \hspace{0.5cm} \text{ } \text{ } 0& -5& \text{ } \text{ }19& 3\alpha-2\beta
			\end{array} \notag
		\end{align} 
		\begin{align}
			\begin{array}{ccc|c}
				\text{ } \hspace{0.3cm} \frac{1}{5}R_1 \hspace{0.75cm} \text{ } \text{ } 1&\text{ } \text{ } 0&
				-\frac{2}{5}& \frac{\alpha+\beta}{5}\\
				\hspace{1.95cm} \text{ } \text{ } 8& \text{ } \text{ } 1&-7& \alpha+2\beta  \\
				\hspace{1.95cm} \text{ } \text{ } 0& -5& \text{ } \text{ }19& 3\alpha-2\beta
			\end{array} \notag
		\end{align} 
		\begin{align}
			\begin{array}{ccc|c}
				\hspace{1.95cm} \text{ } \text{ } 1&\text{ } \text{ } 0& -\frac{2}{5}& \frac{\alpha+\beta}{5}\\
				R_2-8R_1 \hspace{0.25cm} \text{ } \text{ } 0& \text{ } \text{ } 1&-\frac{19}{5}& 
				\frac{2\beta-3\alpha}{5} \\
				\hspace{1.95cm} \text{ } \text{ } 0& -5& \text{ } \text{ }19& 3\alpha-2\beta
			\end{array} \notag
		\end{align} 
		\begin{align}
			\begin{array}{ccc|c}
				\hspace{1.95cm} \text{ } \text{ } 1&\text{ } \text{ } 0& -\frac{2}{5}& \frac{\alpha+\beta}{5}\\
				\hspace{1.95cm} \text{ } \text{ } 0& \text{ } \text{ } 1&-\frac{19}{5}& \frac{2\beta-3\alpha}{5} \\
				-\frac{1}{5}R_3 \text{ } \text{ } \hspace{0.9cm} 0& \text{ } \text{ } 1&-\frac{19}{5}& 
				\frac{2\beta-3\alpha}{5}
			\end{array} \notag
		\end{align} 
		\begin{align}
			\begin{array}{ccc|c}
				\hspace{1.95cm} \text{ } \text{ } 1&\text{ } \text{ } 0& -\frac{2}{5}& \frac{\alpha+\beta}{5}\\
				\hspace{1.95cm} \text{ } \text{ } 0& \text{ } \text{ } 1&-\frac{19}{5}& \frac{2\beta-3\alpha}{5} \\
				R_3-R_2 \hspace{0.5cm} \text{ } \text{ } 0& \text{ } \text{ } 0& \text{ } \text{ } 0& 0
			\end{array} \notag
		\end{align} 
		The last row implies that the equations are not independent and that the planes are intersecting in 			a single straight line. From the first two rows we can write
		\begin{align}
			x-\frac{2}{5}z = \frac{\alpha+\beta}{5} &\Rightarrow x=\frac{2z+\alpha+\beta}{5} \notag \\
			y-\frac{19}{5}z  = \frac{2\beta-3\alpha}{5} &\Rightarrow y=\frac{19z+2\beta-3\alpha}{5} \notag
		\end{align} 
		Thus we have shown that the system requires $\gamma=2\alpha-3\beta$ to solve for two of the 			unknowns, $x$ and $y$, in terms of the third, $z$. Consequently, the system is consistent if and 			only if $\gamma=2\alpha-3\beta$. \\
		
	\item Let $x$, $y$ and $z$ be the number of shares the investor owns in BHP-Billiton, AMP and WMC 
		respectively. Given the changes in both daily share prices and the value  of total shares, the 				following system summarises the data available.
		\begin{align}
			-x-1.5y+0.5z&=-350 \notag \\
			1.5x-0.5y+z&=600 \notag 
		\end{align}
		Thus we have a system of two linear equations in three unknowns. To solve this $2 \times 3$ 				system we use Gauss-Jordan elimination.
		\begin{align}
			\begin{array}{ccc|c}
				\hspace{1.5cm} -1 &-\frac{3}{2} &\frac{1}{2} &-350 \\
				\hspace{1.5cm} \text{ } \text{ }\frac{3}{2} &-\frac{1}{2} & 1 &\text{ } \text{ } 600 
			\end{array} \notag
		\end{align} 
		\begin{align}
			\begin{array}{ccc|c}
				-R_1 \hspace{0.7cm} \text{ } \text{ }1 & \text{ } \text{ } \frac{3}{2} &-\frac{1}{2} &350 \\
				\hspace{1.5cm} \text{ } \text{ }\frac{3}{2} &-\frac{1}{2} & \text{ } \text{ }1 &600 
			\end{array} \notag
		\end{align} 
		\begin{align}
			\begin{array}{ccc|c}
				\hspace{1.9cm} \text{ } \text{ }1 & \text{ } \text{ } \frac{3}{2} &-\frac{1}{2} &350 \\
				R_2-\frac{3}{2}R_1 \hspace{0.5cm} 0 &-\frac{11}{4} & \text{ } \text{ }\frac{7}{4} & \text{ } 75 
			\end{array} \notag
		\end{align} 
		\begin{align}
			\begin{array}{ccc|c}
				\hspace{1.7cm} \text{ } \text{ }1 &\frac{3}{2} &-\frac{1}{2} & \text{ } \text{ } 350 \\
				-\frac{4}{11}R_2 \hspace{0.5cm} \text{ } \text{ } 0 &1 &-\frac{7}{11} &-\frac{300}{11} 
			\end{array} \notag
		\end{align} 
		\begin{align}
			\begin{array}{ccc|c}
				R_1-\frac{3}{2}R_2 \hspace{0.5cm} 1 &0 &\text{ } \text{ } \frac{5}{11} & \text{ } \text{ }
				\frac{4300}{11} \\
				\hspace{1.9cm} \text{ } \text{ } 0 &1 &-\frac{7}{11} &-\frac{300}{11} 
			\end{array} \notag
		\end{align} 
		This is the last row elimination possible and consequently we can solve for only two of the
		unknowns, $x$ and $y$, in terms of the third, $z$.
		\begin{align}
			x+\frac{5}{11}z=\frac{4300}{11} & \Rightarrow x=\frac{4300-5z}{11} \notag \\
			y-\frac{7}{11}z=-\frac{300}{11} & \Rightarrow y=\frac{7z-300}{11} \notag
		\end{align}
		Hence the stockbroker does not have enough information to calculate the number of shares the 			investor owns in each company. \\
		But if the investor tells the stockbroker that $z=200$ then,
		\begin{align}
			x=\frac{4300-5(200)}{11}&=300 \notag \\
			y=\frac{7(200)-300}{11}&=100 \notag
		\end{align}	
	\smallskip				
	\item Given the matrices $A$, $B$ and $C$, only $A-B$ is undefined as a $3\times2$ matrix cannot be
		subtracted from a $3\times3$ matrix. The rest are calculated below.
		\begin{align}
		\hspace{1cm} AB=
			\begin{bmatrix}
				5 &3 &-1 \\
				1 &1 &-1 \\
				3 &0 & \text{ } \text{ } 1
			\end{bmatrix} \cdot
			\begin{bmatrix}
				10 &-10 \\
				\text{ } 3 &-2 \\
				\text{ } 1 &\text{ } \text{ } 0
			\end{bmatrix} =
			\begin{bmatrix}
				50+9-1 &-50-6+0 \\
				10+3-1 &-10-2+0 \\
				30+0+1 &-30+0+0
			\end{bmatrix} =
			\begin{bmatrix}
				58 &-56 \\
				12 &-12 \\
				31 &-30
			\end{bmatrix} \notag
		\end{align}
		
		\begin{align}
		\hspace{0.85cm} AC&=
			\begin{bmatrix}
				5 &3 &-1 \\
				1 &1 &-1 \\
				3 &0 & \text{ } \text{ } 1
			\end{bmatrix} \cdot
			\begin{bmatrix}
				2 &\text{ } \text{ }3 &-1 \\
				0 &-1 &\text{ } \text{ }0 \\
				4 &\text{ } \text{ }2 &-3
			\end{bmatrix}=
			\begin{bmatrix}
				10+0-4 &15-3-2 &-5+0+3 \\
				2+0-4 &3-1-2 &-1+0+3 \\
				6+0+4 &9+0+2 &-3+0-3
			\end{bmatrix}\notag \\
			&=
			\begin{bmatrix}
				\text{ } \text{ }6 &10 &-2 \\
				-2 &\text{ }0 &\text{ } \text{ }2 \\
				\text{ }10 &11 &-6
			\end{bmatrix} \notag
		\end{align}
		
		\begin{align}
		\hspace{1.45cm} CB=
			\begin{bmatrix}
				2 &\text{ } \text{ }3 &-1 \\
				0 &-1 &\text{ } \text{ }0 \\
				4 &\text{ } \text{ }2 &-3
			\end{bmatrix} \cdot
			\begin{bmatrix}
				10 &-10 \\
				\text{ } 3 &-2 \\
				\text{ } 1 &\text{ } \text{ } 0
			\end{bmatrix} =
			\begin{bmatrix}
				20+9-1 &-20-6+0 \\
				\text{ }0-3+0 &\text{ } \text{ } \text{ } \text{ }0+2+0 \\
				40+6-3 &-40-4+0
			\end{bmatrix} =
			\begin{bmatrix}
				\text{ }28 &-26 \\
				-3 &\text{ } \text{ } \text{ }2 \\
				\text{ }43 &-44
			\end{bmatrix} \notag
		\end{align}
		
		\begin{align}
			A+3C=
			\begin{bmatrix}
				5 &3 &-1 \\
				1 &1 &-1 \\
				3 &0 & \text{ } \text{ } 1
			\end{bmatrix} +
			\begin{bmatrix}
				6 &\text{ } \text{ }9 &-3 \\
				0 &-3 &\text{ } \text{ }0 \\
				12 &\text{ } \text{ }6 &-9
			\end{bmatrix} =
			\begin{bmatrix}
				11 &\text{ }12 &-4 \\
				\text{ } \text{ }1 &-2 &-1 \\
				15 &\text{ } \text{ }6 &-8
			\end{bmatrix} \notag
		\end{align}
		
		\begin{align}
			\hspace{0.9cm} AC-CA&=
			\begin{bmatrix}
				\text{ } \text{ }6 &10 &-2 \\
				-2 &\text{ }0 &\text{ } \text{ }2 \\
				\text{ }10 &11 &-6
			\end{bmatrix} -
			\begin{bmatrix}
				2 &\text{ } \text{ }3 &-1 \\
				0 &-1 &\text{ } \text{ }0 \\
				4 &\text{ } \text{ }2 &-3
			\end{bmatrix} \cdot
			\begin{bmatrix}
				5 &3 &-1 \\
				1 &1 &-1 \\
				3 &0 & \text{ } \text{ } 1
			\end{bmatrix} \notag \\
			&=
			\begin{bmatrix}
				\text{ } \text{ }6 &10 &-2 \\
				-2 &\text{ }0 &\text{ } \text{ }2 \\
				\text{ }10 &11 &-6
			\end{bmatrix} -
			\begin{bmatrix}
				10+3-3 &\text{ }6+3+0 &-2-3-1 \\
				\text{ } \text{ }0-1+0 &\text{ }0-1+0 &\text{ } \text{ }0+1+0 \\
				20+2-9 &12+2+0 &-4-2-3
			\end{bmatrix} \notag \\
			&=
			\begin{bmatrix}
				\text{ } \text{ }6 &10 &-2 \\
				-2 &\text{ }0 &\text{ } \text{ }2 \\
				\text{ }10 &11 &-6
			\end{bmatrix} -
			\begin{bmatrix}
				\text{ }10 &\text{ } \text{ }9 &-6 \\
				-1 &-1 &\text{ } \text{ }1 \\
				\text{ }13 &\text{ }14 &-9
			\end{bmatrix} =
			\begin{bmatrix}
				-4 &\text{ } \text{ }1 &4 \\
				-1 &\text{ } \text{ }1 &1 \\
				-3 &-3 & 3 
			\end{bmatrix} \notag
		\end{align}
		
	\item Since $A$ is a $2\times2$ matrix and $\det A \ne 0$,
		\begin{align}
			A^{-1}=
			\frac{1}{2\times2-1\times2}
			\begin{bmatrix}
				\text{ } \text{ }2 &-1 \\
				-2 &\text{ } \text{ }2
			\end{bmatrix} =
			\frac{1}{2}\begin{bmatrix}
				\text{ } \text{ }2 &-1 \\
				-2 &\text{ } \text{ }2
			\end{bmatrix} =
			\begin{bmatrix}
				\text{ } \text{ }1 &-\frac{1}{2} \\
				-1 &\text{ } \text{ }1
			\end{bmatrix} \notag
		\end{align}
		To find $B^{-1}$ we consider the following augmented matrix and perform the row reductions 
		below. Note that we are simply solving the system given by $BB^{-1}=I$.	
		\begin{align}
			\begin{array}{ccc|ccc}
				\hspace{1.6cm} 1 &2 &3 &1 &0 &0 \\
				\hspace{1.6cm} 1 &1 &2 &0 &1 &0  \\
				\hspace{1.6cm} 0 &1&2 &0 &0 &1
			\end{array} \notag
		\end{align} 
		\begin{align}
			\begin{array}{ccc|ccc}
				R_1-2R_3 \hspace{0.5cm} 1 &0 &-1 &1 &0 &-2 \\
				\hspace{2.2cm} 1 &1 &\text{ } \text{ }2 &0 &1 &\text{ } \text{ }0  \\
				\hspace{2.2cm} 0& 1&\text{ } \text{ }2 &0 &0 &\text{ } \text{ }1
			\end{array} \notag
		\end{align} 
		\begin{align}
			\begin{array}{ccc|ccc}
				\hspace{2.4cm} 1 &0 &-1 &\text{ } \text{ }1 &0 &-2 \\
				\hspace{0.3cm} R_2-R_1 \hspace{0.6cm} 0 &1 &\text{ } \text{ }3 &-1 &1 &\text{ } \text{ }2  \\
				\hspace{2.4cm} 0& 1&\text{ } \text{ }2 &\text{ } \text{ }0 &0 &\text{ } \text{ }1
			\end{array} \notag
		\end{align} 
		\begin{align}
			\begin{array}{ccc|ccc}
				\hspace{2.4cm} 1 &0 &-1 &\text{ } \text{ }1 &\text{ } \text{ }0 &-2 \\
				\hspace{2.4cm} 0 &1 &\text{ } \text{ }3 &-1 &\text{ } \text{ }1 &\text{ } \text{ }2  \\
				\hspace{0.3cm} R_3-R_2 \hspace{0.6cm} 0& 0&-1 &\text{ } \text{ }1 &-1 &-1
			\end{array} \notag
		\end{align} 
		\begin{align}
			\begin{array}{ccc|ccc}
				\hspace{0.3cm} R_1-R_3 \hspace{0.6cm} 1 &0 &\text{ } \text{ }0 &\text{ } \text{ }0 &\text{ } 					\text{ }1 &-1 \\
				\hspace{2.4cm} 0 &1 &\text{ } \text{ }3 &-1 &\text{ } \text{ }1 &\text{ } \text{ }2  \\
				\hspace{2.4cm} 0& 0&-1 &\text{ } \text{ }1 &-1 &-1
			\end{array} \notag
		\end{align} 
		\begin{align}
			\begin{array}{ccc|ccc}
				\hspace{2.4cm} 1 &0 &0 &\text{ } \text{ }0 &1 &-1 \\
				\hspace{2.4cm} 0 &1 &3 &-1 &1 &\text{ } \text{ }2  \\
				-R_3 \hspace{1.6cm} 0& 0&1 &-1 &1 &\text{ } \text{ }1
			\end{array} \notag
		\end{align} 
		\begin{align}
			\begin{array}{ccc|ccc}
				\hspace{2.4cm} 1 &0 &0 &\text{ } \text{ }0 &\text{ } \text{ }1 &-1 \\
				\hspace{0.1cm} R_2-3R_3 \hspace{0.6cm} 0 &1 &0 &\text{ } \text{ }2 &-2 &-1 \\
				\hspace{2.4cm} 0& 0&1 &-1 &\text{ } \text{ }1 &\text{ } \text{ }1
			\end{array} \notag
		\end{align} 
		Thus,
		\begin{align}
			B^{-1}=
			\begin{bmatrix}
				\text{ } \text{ }0 &\text{ } \text{ }1 &-1 \text{ } \\
				\text{ } \text{ }2 &-2 &-1 \text{ } \\
				-1 &\text{ } \text{ }1 &\text{ } \text{ }1 \text{ }
			\end{bmatrix} \notag
		\end{align} \\
		We perform the same procedure on following system to find $C^{-1}$.
		\begin{align}
			\begin{array}{ccc|ccc}
				\hspace{1.6cm} 3 &\text{ } \text{ }1 &0 &1 &0 &0 \\
				\hspace{1.6cm} 1 &-1 &2 &0 &1 &0  \\
				\hspace{1.6cm} 1 &\text{ } \text{ }1 &1 &0 &0 &1
			\end{array} \notag
		\end{align} 
		\begin{align}
			\begin{array}{ccc|ccc}
				\hspace{2cm} 3 &1 &\text{ } \text{ }0 &1 &\text{ } \text{ }0 &0 \\
				\hspace{0.1cm} -R_2\hspace{0.55cm} -1 &1 &-2 &0 &-1 &0  \\
				\hspace{2cm} 1 &1 &\text{ } \text{ }1 &0 &\text{ } \text{ }0 &1
			\end{array} \notag
		\end{align} 
		\begin{align}
			\begin{array}{ccc|ccc}
				R_1-2R_3 \hspace{0.6cm} 1 &-1 &-2 &1 &\text{ } \text{ }0 &-2 \\
				\hspace{2cm} -1 &\text{ } \text{ }1 &-2 &0 &-1 &\text{ } \text{ }0  \\
				\hspace{2.3cm} 1 &\text{ } \text{ }1 &\text{ } \text{ }1 &0 &\text{ } \text{ }0 &\text{ } \text{ }1
			\end{array} \notag
		\end{align} 
		\begin{align}
			\begin{array}{ccc|ccc}
				\hspace{2.3cm} 1 &-1 &-2 &\text{ } \text{ }1 &\text{ } \text{ }0 &-2 \\
				\hspace{2cm} -1 &\text{ } \text{ }1 &-2 &\text{ } \text{ }0 &-1 &\text{ } \text{ }0  \\
				\hspace{0.2cm} R_3-R_1 \hspace{0.65cm} 0 &\text{ } \text{ }2 &\text{ } \text{ }3 &-1 &\text{ } 				\text{ }0 &\text{ } \text{ }3
			\end{array} \notag
		\end{align} 
		\begin{align}
			\begin{array}{ccc|ccc}
				R_1+\frac{1}{2}R_3 \hspace{0.6cm} 1 &0 &-\frac{1}{2} &\text{ } \text{ }\frac{1}{2} &\text{ } 					\text{ }0 &-\frac{1}{2} \\
				\hspace{2cm} -1 &1 &-2 &\text{ } \text{ }0 &-1 &\text{ } \text{ }0  \\
				\hspace{2.3cm} 0 &2 &\text{ } \text{ }3 &-1 &\text{ } \text{ }0 &\text{ } \text{ }3
			\end{array} \notag
		\end{align} 
		\begin{align}
			\begin{array}{ccc|ccc}
				\hspace{2.3cm} 1 &0 &-\frac{1}{2} &\text{ } \text{ }\frac{1}{2} &\text{ } \text{ }0 &-\frac{1}{2} \\
				R_2+R_1\hspace{0.8cm} 0 &1 &-\frac{5}{2} &\text{ } \text{ }\frac{1}{2} &-1 
				&-\frac{1}{2} \\
				\hspace{2.3cm} 0 &2 &\text{ } \text{ }3 &-1 &\text{ } \text{ }0 &\text{ } \text{ }3
			\end{array} \notag
		\end{align} 
		\begin{align}
			\begin{array}{ccc|ccc}
				\hspace{2.8cm} 1 &0 &-\frac{1}{2} &\text{ } \text{ }\frac{1}{2} &\text{ } \text{ }0 					&-\frac{1}{2} \\
				\hspace{2.8cm} 0 &1 &-\frac{5}{2} &\text{ } \text{ }\frac{1}{2} &-1 
				&-\frac{1}{2} \\
				\frac{1}{8}(R_3-2R_2) \hspace{0.6cm} 0 &0 &\text{ } \text{ }1 &-\frac{1}{4} &\text{ } \text{ }					\frac{1}{4} &\text{ } \text{ }\frac{1}{2}
			\end{array} \notag
		\end{align} 
		\begin{align}
			\begin{array}{ccc|ccc}
				R_1-\frac{1}{2}R_3 \hspace{0.6cm} 1 &0 &0 &\text{ } \text{ }\frac{3}{8} &\text{ } \text{ }
				\frac{1}{8} &-\frac{1}{4} \\
				R_2+\frac{5}{2}R_3 \hspace{0.6cm} 0 &1 &0 &-\frac{1}{8} &-\frac{3}{8} &\text{ } \text{ }
				\frac{3}{4} \\
				\hspace{2.35cm} 0 &0 &1 &-\frac{1}{4} &\text{ } \text{ }\frac{1}{4} &\text{ } \text{ }\frac{1}{2}
			\end{array} \notag
		\end{align} 
		Thus,
		\begin{align}
			C^{-1}=
			\begin{bmatrix}
				\text{ } \text{ } \frac{3}{8} &\text{ } \text{ } \frac{1}{8} &-\frac{1}{4} \text{ } \\
				-\frac{1}{8} &-\frac{3}{8} &\text{ } \text{ } \frac{3}{4} \text{ } \\
				-\frac{1}{4} &\text{ } \text{ }\frac{1}{4} &\text{ } \text{ }\frac{1}{2} \text{ } 
			\end{bmatrix} = \frac{1}{8}
			\begin{bmatrix}
				\text{ } \text{ }3 &\text{ } \text{ }1 &-2 \text{ } \\
				-1 &-3 &\text{ } \text{ } 6 \text{ } \\
				-2 &\text{ } \text{ }2 &\text{ } \text{ }4 \text{ } 
			\end{bmatrix} \notag
		\end{align}		
		\\
	\item We calculate the determinant by an expansion on the first row where
		\begin{align}
			\det=c_{11}\times3+c_{12}\times0+c_{13}\times2 \notag
		\end{align}
		\begin{align}
			c_{11}&=(-1)^{1+1}
			\begin{vmatrix}
				\text{ }1 &1 \text{ } \\
				\text{ }1 &4 \text{ }
			\end{vmatrix} = 1(1\times4-1\times1)=3 \notag \\
			c_{13}&=(-1)^{1+3}
			\begin{vmatrix}
				-1 &1 \text{ } \\
				\text{ } \text{ }0 &1 \text{ }
			\end{vmatrix} = 1(-1\times1-1\times0)=-1 \notag
		\end{align}
		\begin{align}
			\Rightarrow \det=3\times3+0-1\times2=7 \notag
		\end{align} \\
		Expanding by the third row we find
		\begin{align}
			\det=c_{31}\times0+c_{32}\times1+c_{33}\times4 \notag
		\end{align}
		\begin{align}
			c_{32}&=(-1)^{3+2}
			\begin{vmatrix}
				\text{ } \text3 &2 \text{ } \\
				\-1 &1 \text{ }
			\end{vmatrix} = -1(3\times1+2\times1)=-5 \notag \\
			c_{33}&=(-1)^{3+3}
			\begin{vmatrix}
				\text{ } \text{ }3 &0 \text{ } \\
				-1 &1 \text{ }
			\end{vmatrix} = 1(3\times1+0\times1)=3\notag
		\end{align}
		\begin{align}
			\Rightarrow \det=0-5\times1+3\times4=7 \notag
		\end{align} \\
		Thus we have verified that the two expansions yield the same determinant.
		\\
		\item We construct the following system to find an interpolating polynomial of the form $p(x)=a_0
		+a_1x+a_2x^2$ for the given data points.
		\begin{align}
			\begin{bmatrix}
				1 &1 &1 \\
				1 &2 &4 \\
				1 &3 &9 
			\end{bmatrix} \cdot
			\begin{bmatrix}
				a_0 \\
				a_1 \\
				a_2
			\end{bmatrix} =
			\begin{bmatrix}
				12 \\
				15 \\
				16
			\end{bmatrix}  \notag
		\end{align} \\
		We then solve for the unknown coefficients $a_0, a_1$ and $a_2$ using Cramer's Rule. Let 
		\begin{align}
			A=
			\begin{bmatrix}
				1 &1 &1 \\
				1 &2 &4 \\
				1 &3 &9 
			\end{bmatrix} \hspace{0.3cm} \text{and} \hspace{0.3cm} b=
			\begin{bmatrix}
				12 \\
				15 \\
				16
			\end{bmatrix}  \notag
		\end{align}
		Then,
		\begin{align}
		A_0=
			\begin{bmatrix}
				12 &1 &1 \\
				15 &2 &4 \\
				16 &3 &9 
			\end{bmatrix}, \hspace{0.3cm} A_1=
			\begin{bmatrix}
				1 &12 &1 \\
				1 &15 &4 \\
				1 &16 &9 
			\end{bmatrix}, \hspace{0.3cm} A_2=
			\begin{bmatrix}
				1 &1 &12 \\
				1 &2 &15 \\
				1 &3 &16
			\end{bmatrix} \notag
		\end{align} \\
		The determinants of each $3\times3$ matrix are given below. \\
		$$|A|=2 \text{ } (\ne0), \text{ } |A_0|=14,  \text{ } |A_1|=12, \text{ } |A_2|=-2$$
		Hence,
		\begin{align}
			a_0&=\frac{|A_0|}{|A|}=\frac{14}{2}=7 \notag \\
			a_1&=\frac{|A_1|}{|A|}=\frac{12}{2}=6 \notag \\
			a_2&=\frac{|A_2|}{|A|}=-\frac{2}{2}=-1 \notag
		\end{align} \\
		The interpolating polynomial is therefore
		$$p(x)=7+6x-x^2$$
		
\end{enumerate}
	
\end{document}