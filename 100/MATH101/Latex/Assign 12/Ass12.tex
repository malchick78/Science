\documentclass[12pt]{amsart}

\usepackage{a4wide, amsxtra}

\usepackage[pdftex]{graphicx}

\usepackage{hyperref}

 \title{MATH101 Assignment 12}

 \author{Mark Villar}

\begin{document} 

\maketitle 

\begin{enumerate}
	
	\item After converting each matrix to  triangular form we use the properties of triangular matrices to 				evaluate the following determinants.   	
		\begin{enumerate}
			\item 
			\begin{align}
				\begin{array}{cccc}
					\hspace{1.8cm} \text{ } \text{ }2 &0 &-1 &\text{ } \text{ }0 \\
					R_2+R_4 \hspace{0.3cm} \text{ } \text{ }1 &5 &\text{ } \text{ }3 &\text{ } \text{ }0 \\
					\hspace{1.7cm} -2 &3 &\text{ } \text{ }1 &\text{ } \text{ }0 \\
					\hspace{1.8cm} \text{ } \text{ }0 &4 &\text{ } \text{ }2 &-1
				\end{array} \Rightarrow
				\begin{array}{cccc}
					\hspace{2cm} \text{ } \text{ }2 &0 &-1 &\text{ } \text{ }0 \\
					R_2+3R_1 \hspace{0.3cm} \text{ } \text{ }7 &5 &\text{ } \text{ }0 &\text{ } \text{ }0 \\
					\hspace{2cm} -2 &3 &\text{ } \text{ }1 &\text{ } \text{ }0 \\
					\hspace{2cm} \text{ } \text{ }0 &4 &\text{ } \text{ }2 &-1
				\end{array} \notag \\
				\begin{array}{cccc}
					R_1+R_3 \hspace{0.5cm} \text{ } \text{ }0 &3 &0 &\text{ } \text{ }0 \\
					\hspace{2cm} \text{ } \text{ }7 &5 &0 &\text{ } \text{ }0 \\
					\hspace{2cm} -2 &3 &1 &\text{ } \text{ }0 \\
					\hspace{2cm} \text{ } \text{ }0 &4 &2 &-1
				\end{array} \Rightarrow
				\begin{array}{cccc}
					R_1-\frac{3}{5}R_2 \hspace{0.3cm} -\frac{21}{5}&0 &0 &\text{ } \text{ }0 \\
					\hspace{2.2cm} \text{ } \text{ }7 &5 &0 &\text{ } \text{ }0 \\
					\hspace{2.2cm} -2 &3 &1 &\text{ } \text{ }0 \\
					\hspace{2.2cm} \text{ } \text{ }0 &4 &2 &-1
				\end{array} \notag
			\end{align} \\
			Thus we have a lower triangular matrix with determinant equal to the product of its main 					diagonal elements. It follows that \\
			\begin{align}
				\hspace{1.5cm}
				\begin{vmatrix}
					\text{ } \text{ }2 &0 &-1 &\text{ } \text{ }0 \text{ } \\
					\text{ } \text{ }1 &1 &\text{ } \text{ }1 &\text{ } \text{ }1 \text{ }  \\
					-2 &3 &\text{ } \text{ }1 &\text{ } \text{ }0 \text{ } \\
					\text{ } \text{ }0 &4 &\text{ } \text{ }2 &-1 \text{ } 
				\end{vmatrix} = 
				\begin{vmatrix}
					-\frac{21}{5}&0 &0 &\text{ } \text{ }0 \text{ } \\
					\text{ } \text{ }7 &5 &0 &\text{ } \text{ }0 \text{ } \\
					-2 &3 &1 &\text{ } \text{ }0 \text{ } \\
					\text{ } \text{ }0 &4 &2 &-1 \text{ } 
				\end{vmatrix} \notag
			\end{align}
			\begin{align}
				\hspace{1cm}=-\frac{21}{5}\times5\times1\times(-1)=21 \notag
			\end{align}
			
			\item 
			\begin{align}
				\begin{array}{cccc}
					\hspace{2.2cm} -1 &\text{ } \text{ }3 &\text{ } \text{ } \text{ }2 &\text{ } \text{ }1 \\
					\hspace{2.2cm} -7 &-6 &\text{ } \text{ } \text{ }0 &-2 \\
					R_3-7R_1 \hspace{0.3cm} -3 &\text{ } \text{ }0 &-16 &\text{ } \text{ }0 \\
					\hspace{2.2cm} \text{ } \text{ }1 &\text{ } \text{ }9 &\hspace{0.15cm}-5 &\text{ } \text{ }3
				\end{array} \Rightarrow
				\begin{array}{cccc}
					R_1+\frac{1}{2}R_2 \hspace{0.3cm} \text{ } \text{ } \text{ } \text{ } \frac{5}{2} 
					&\text{ } \text{ }0 & \text{ } \text{ } \text{ }2 &\text{ } \text{ }0 \\
					\hspace{2.2cm} -7 &-6 &\text{ } \text{ } \text{ }0 &-2 \\
					\hspace{2.2cm} -3 &\text{ } \text{ }0 &-16 &\text{ } \text{ }0 \\
					\hspace{2.2cm} \text{ } \text{ }1 &\text{ } \text{ }9 &\hspace{0.15cm}-5 &\text{ } \text{ }3
				\end{array} \notag \\
				\begin{array}{cccc}
					R_1+\frac{1}{8}R_3 \hspace{0.3cm} \text{ } \text{ } \text{ } \text{ } \frac{17}{8} 
					&\text{ } \text{ }0 & \text{ } \text{ } \text{ }0 &\text{ } \text{ }0 \\
					\hspace{2.2cm} -7 &-6 &\text{ } \text{ } \text{ }0 &-2 \\
					\hspace{2.2cm} -3 &\text{ } \text{ }0 &-16 &\text{ } \text{ }0 \\
					\hspace{2.2cm} \text{ } \text{ }1 &\text{ } \text{ }9 &\hspace{0.15cm}-5 &\text{ } \text{ }3
				\end{array} \Rightarrow
				\begin{array}{cccc}
					\hspace{2.3cm} \text{ } \text{ } \frac{17}{8} &0 & \text{ } \text{ }\text{ }0 &0 \\
					R_2+\frac{2}{3}R_4 \hspace{0.3cm} -\frac{19}{3} &0 &-\frac{10}{3} &0 \\
					\hspace{2.2cm} -3 &0 &-16 &0 \\
					\hspace{2.2cm} \text{ } \text{ }1 &9 &-5 &3
				\end{array} \notag
			\end{align}
			\begin{align}
				\Rightarrow
				\begin{array}{cccc}
					\hspace{2.5cm} \text{ } \text{ } \frac{17}{8} &0 &\text{ } \text{ }0 &0 \\
					R_2-\frac{5}{24}R_3 \hspace{0.3cm} -\frac{137}{24} &0 &\text{ } \text{ }0 &0 \\
					\hspace{2.4cm} -3 &0 &-16 &0 \\
					\hspace{2.4cm} \text{ } \text{ }1 &9 &-5 &3
				\end{array} \notag
			\end{align}
			\\
			Again we have a lower triangular matrix so 
			\begin{align}
				\begin{vmatrix}
					-1 &\text{ } \text{ }3 &\text{ } \text{ } \text{ }2 &\text{ } \text{ }1 \text{ } \\
					-7 &-6 &\text{ } \text{ } \text{ }0 &-2 \text{ } \\
					-10 &\text{ }21 &-2 &\text{ } \text{ }7 \text{ } \\
					\text{ } \text{ } \text{ }1 &\text{ } \text{ }9 &-5 &\text{ } \text{ }3 \text{ }
				\end{vmatrix}=
				\begin{vmatrix}
					\text{ } \text{ } \frac{17}{8} &0 &\text{ } \text{ }0 &0 \text{ } \\
					-\frac{137}{24} &0 &\text{ } \text{ }0 &0 \text{ } \\
					-3 &0 &-16 &0 \text{ } \\
					\text{ } \text{ }1 &9 &\hspace{0.15cm}-5 &3 \text{ }
				\end{vmatrix} =0 \notag
			\end{align}
			as one of the main diagonal elements is $0$.
			\medskip
			\item Clearly we have an upper triangular matrix so its determinant is simply
			\begin{align}
				\begin{vmatrix}
					-9 &12 &27 &\text{ } \text{ }1 \text{ } \\
					\text{ } \text{ }0 &\text{ }2 &\text{ }0 &\text{ }15 \text{ } \\
					\text{ } \text{ }0 &\text{ }0 &\text{ }1 &\text{ }10 \text{ } \\
					\text{ } \text{ }0 &\text{ }0 &\text{ }0 &-2 \text{ }
				\end{vmatrix}=-9\times2\times1\times(-2)=36 \notag
			\end{align}
			
		\end{enumerate}
		
	\item Expansions on each row are shown below.
			\begin{align}
				\begin{vmatrix}
					\text{ }a &b &c \text{ } \\
					\text{ }d &e &f \text{ } \\
					\text{ }g &h &j \text{ }
				\end{vmatrix}=7
				&=a(ej-fh)-b(dj-fg)+c(dh-eg) \notag \\
				&=-d(bj-ch)+e(aj-cg)-f(ah-bg) \notag \\
				&=g(bf-ce)-h(af-cd)+j(ae-bd) \notag
			\end{align}

		\begin{enumerate}
		
			\item Expanding on the third row,
			\begin{align}
				\begin{vmatrix}
					\text{ }a &b &c \text{ } \\
					\text{ }d &e &f \text{ } \\
					\text{ }5g &5h &5j \text{ }
				\end{vmatrix}&=5g(bf-ce)-5h(af-cd)+5j(ae-bd) \notag \\
				&=5\big(g(bf-ce)-h(af-cd)+j(ae-bd)\big) \notag \\
				&=5\times7=35 \notag
			\end{align}
			
			\item Expanding on the second row,
			\begin{align}
				\hspace{0.8cm}
				\begin{vmatrix}
					a &b &c \\
					\text{ }2d+a &2e+b &2f+c \text{ } \\
					g &h &j
				\end{vmatrix}&=-(2d+a)(bj-ch)+(2e+b)(aj-cg)-(2f+c)(ah-bg) \notag \\
				&=2(-d(bj-ch)+e(aj-cg)-f(ah-bg))-a(bj-ch) \notag \\
				&\hspace{0.4cm} +b(aj-cg)-c(ah-bg) \notag \\
				&=2\times7-a(bj-ch)+b(aj-cg)-c(ah-bg) \notag
			\end{align}
			We then find the value of the three remaining terms by evaluating the following determinant.
			Expansions on the second and third rows are shown below.
			\begin{align}
				\begin{vmatrix}
					\text{ }a &b &c \text{ } \\
					\text{ }a &b &c \text{ } \\
					\text{ }g &h &j \text{ }
				\end{vmatrix}&=-a(bj-ch)+b(aj-cg)-c(ah-bg) \notag \\
				&=g(bc-bc)-h(ac-ac)+j(ab-ab)=0 \notag
				\notag
				\end{align}
			Thus,
			\begin{align}
				\begin{vmatrix}
					a &b &c \\
					\text{ }2d+a &2e+b &2f+c \text{ } \\
					g &h &j
				\end{vmatrix} =2\times7+0=14 \notag
			\end{align}
			
			\item We need only expand by the first row to confirm that the determinant remains the same. 
				The other two row expansions, which are simply the negatives of the original expansions
				above, also confirm that the determinant does not change.
			\begin{align}
				\begin{vmatrix}
					\text{ }g &h &j \text{ } \\
					\text{ }a &b &c \text{ } \\
					\text{ }d &e &f \text{ }
				\end{vmatrix}
				&=g(bf-ce)-h(af-cd)+j(ae-bd)=7  \notag \\
				&=-a(fh-ej)+b(fg-dj)-c(eg-dh) \notag \\
				&=d(ch-bj)-e(cg-aj)+f(bg-ah) \notag \\		
				&=a(ej-fh)-b(dj-fg)+c(dh-eg) \notag \\
				&=-d(bj-ch)+e(aj-cg)-f(ah-bg) \notag	
			\end{align}

		\end{enumerate}
								
	\item 
	
		\begin{enumerate}
		
			\item 
			\begin{align}
				\begin{vmatrix}
					\text{ }\text{ }10 &1 \text{ } \\
					-20 &1 \text{ } 
				\end{vmatrix}=\frac{1}{10\times1-(-20)\times1}=\frac{1}{10+20} =\frac{1}{30} \ne 0 \notag
			\end{align} \\
			Since the matrix is square with determinant not equal to zero it is invertible.  Its inverse is 
			simply
			\begin{align}
				\begin{pmatrix}
					\text{ }\text{ }10 &1 \\
					-20 &1
				\end{pmatrix}^{-1}=
				\frac{1}{30} 
				\begin{pmatrix}
					1 &-1 \\
					20 &\text{ } 10
				\end{pmatrix}=
				\begin{pmatrix}
					\frac{1}{30} &-\frac{1}{30} \\
					\frac{2}{3} &\text{ } \text{ } \frac{1}{3}
				\end{pmatrix} \notag
			\end{align}
																
			\item Expanding on the second row,
			\begin{align}
				\hspace{2cm} 
				\begin{vmatrix}
					\text{ }1 &0 &5 \text{ } \\
					\text{ }1 &1 &0 \text{ }  \\
					\text{ }3 &2 &6 \text{ } 
				\end{vmatrix}=-1(0\times6-5\times2)+1(1\times6-5\times3)+0=10-9=1 \ne 0 \notag
			\end{align}
			This matrix is also invertible as it is square with determinant not equal to $0$. 
			\\
			\\
			To find its inverse we simply need to calculate the adjoint as the determinant is $1$. We 					transpose the matrix of cofactors below to find the adjoint.
			\begin{align}
			\hspace{1.5cm} \text{adj}&=
				\begin{pmatrix}
					\text{ } \text{ }1\times6-0\times2 &-(1\times6-0\times3) &\text{ } \text{ } 
					1\times2-1\times3 \\
					-(0\times6-5\times2) &\text{ } \text{ }1\times6-5\times3 &-(1\times2-0\times3)\\
					\text{ } \text{ }0\times0-5\times1 &-(1\times0-5\times1) &\text{ } \text{ } 
					1\times1-0\times1
				\end{pmatrix}^T \notag \\
				&= 
				\begin{pmatrix}
					\text{ } \text{ }6 &-6 &-1 \text{ } \\
					\text{ }10 &-9 &-2 \text{ } \\
					-5 &\text{ } \text{ }5 &\text{ } \text{ }1 \text{ }
				\end{pmatrix}^T=
				\begin{pmatrix}
					\text{ } \text{ }6 &\text{ }10 &-5 \text{ } \\
					-6 &-9 &\text{ } \text{ }5 \text{ } \\
					-1 &-2 &\text{ } \text{ }1 \text{ }
				\end{pmatrix}
				\notag
			\end{align}
			Hence,
			\begin{align}
				\begin{pmatrix}
					\text{ }1 &0 &5 \text{ } \\
					\text{ }1 &1 &0 \text{ }  \\
					\text{ }3 &2 &6 \text{ } 
				\end{pmatrix}^{-1}=
				\begin{pmatrix}
					\text{ } \text{ }6 &\text{ }10 &-5 \text{ } \\
					-6 &-9 &\text{ } \text{ }5 \text{ } \\
					-1 &-2 &\text{ } \text{ }1 \text{ }
				\end{pmatrix} \notag
			\end{align}
			\smallskip
			\item Using the following row eliminations we convert this matrix to lower triangular form as we 			did in Question 1.
			$$R_2-R_1, \hspace{0.5cm} R_3+2R_1, \hspace{0.5cm} R_2+\frac{2}{3}R_3,$$
			$$R_1+\frac{3}{2}R_2, \hspace{0.5cm} R_1-\frac{1}{3}R_3,$$
			$$R_1+\frac{5}{6}R_2,  \hspace{0.5cm} R_1+\frac{8}{9}R_3$$
				\\
			Thus the product of its diagonal elements is our determinant.
			\begin{align}
				&\begin{vmatrix}
					\text{ } \text{ }1 &\text{ } \text{ }3 &\text{ } \text{ }0 &-1 \text{ } \\
					\text{ } \text{ }0 &\text{ } \text{ }1 &-2 &-1 \text{ }  \\
					-2 &-6 &\text{ } \text{ }3 &\text{ } \text{ }2 \text{ } \\
					\text{ } \text{ }3 &\text{ } \text{ }5 &\text{ } \text{ }8 &-3 \text{ } 
				\end{vmatrix}=
				\begin{vmatrix}
					-\frac{2}{3} &\text{ } \text{ }0 &0 & \text{ } \text{ } 0 \text{ }\\
					-1 &-2 &0 &\text{ } \text{ }0 \text{ }  \\
					\text{ } \text{ }0 &\text{ } \text{ }0 &3 &\text{ } \text{ }0 \text{ } \\
					\text{ } \text{ }3 &\text{ } \text{ }5 &8 &-3 \text{ } 
				\end{vmatrix} \notag \\
				&=-\frac{2}{3}\times(-2)\times3\times(-3)=-12 \ne 0 \notag
			\end{align}
			
			Hence the matrix is invertible and we calculate its inverse as follows.
			\begin{align}
			c_{11}=
				\begin{vmatrix}
					\text{ } \text{ }1 &-2 &-1 \text{ } \\
					-6 &\text{ } \text{ }3 &\text{ } \text{ }2 \text{ } \\
					\text{ } \text{ }5 &\text{ } \text{ }8 &-3 \text{ } \\
				\end{vmatrix}=& \text{ } 1(3\times(-3)-2\times8)-(-2)(-6\times(-3)-2\times5) \notag \\
				&+(-1)(-6\times8-3\times5) = -25+16+63=54 \notag
			\end{align}
			\begin{align}
			\hspace{1cm} c_{12}=-
				\begin{vmatrix}
					\text{ } \text{ }0 &-2 &-1 \text{ } \\
					-2 &\text{ } \text{ }3 &\text{ } \text{ }2 \text{ } \\
					\text{ } \text{ }3 &\text{ } \text{ }8 &-3 \text{ } \\
				\end{vmatrix}&=-\big[0-(-2)(-2\times(-3)-2\times3)+(-1)(-2\times8-3\times3)\big] \notag \\
				&=-\big[0+0+25\big]=-25 \notag
			\end{align}
			\begin{align}
			\hspace{0.9cm} c_{13}=
				\begin{vmatrix}
					\text{ } \text{ }0 &\text{ } \text{ }1 &-1 \text{ } \\
					-2 &-6 &\text{ } \text{ }2 \text{ } \\
					\text{ } \text{ }3 &\text{ } \text{ }5 &-3 \text{ } \\
				\end{vmatrix}&=0-1(-2\times(-3)-2\times3)+(-1)(-2\times5-(-6)\times3) \notag \\
				&=0+0-8=-8 \notag
			\end{align}
			\begin{align}
			\hspace{1cm} c_{14}=-
				\begin{vmatrix}
					\text{ } \text{ }0 &\text{ } \text{ }1 &-2 \text{ } \\
					-2 &-6 &\text{ } \text{ }3 \text{ } \\
					\text{ } \text{ }3 &\text{ } \text{ }5 &\text{ } \text{ }8 \text{ } \\
				\end{vmatrix}&=-\big[0-1(-2\times8-3\times3)+(-2)(-2\times5-(-6)\times3)\big] \notag \\
				&=-\big[0+25-16\big]=-9 \notag
			\end{align}
			\begin{align}
			\hspace{0.8cm} c_{21}=-
				\begin{vmatrix}
					\text{ } \text{ }3 &\text{ } \text{ }0 &-1 \text{ } \\
					-6 &\text{ } \text{ }3 &\text{ } \text{ }2 \text{ } \\
					\text{ } \text{ }5 &\text{ } \text{ }8 &-3 \text{ } \\
				\end{vmatrix}&=-\big[3(3\times(-3)-2\times8)-0+(-1)(-6\times8-3\times5)\big] \notag \\
				&=-\left[-75+0+63\right]=12 \notag
			\end{align}
			\begin{align}
			c_{22}=
				\begin{vmatrix}
					\text{ } \text{ }1 &\text{ } \text{ }0 &-1 \text{ } \\
					-2 &\text{ } \text{ }3 &\text{ } \text{ }2 \text{ } \\
					\text{ } \text{ }3 &\text{ } \text{ }8 &-3 \text{ } \\
				\end{vmatrix}&=1(3\times(-3)-2\times8)-0+(-1)(-2\times8-3\times3) \notag \\
				&=-25+0+25=0 \notag
			\end{align}
			\begin{align}
			\hspace{0.4cm} c_{23}=-
				\begin{vmatrix}
					\text{ } \text{ }1 &\text{ } \text{ }3 &-1 \text{ } \\
					-2 &-6 &\text{ } \text{ }2 \text{ } \\
					\text{ } \text{ }3 &\text{ } \text{ }5 &-3 \text{ } \\
				\end{vmatrix}&=-\big[1(-6\times(-3)-2\times5)-3(-2\times(-3)-2\times3) \notag \\
				&\hspace{0.45cm}+(-1)(-2\times5-(-6)\times3)\big]=-\big[8+0-8\big]=0 \notag 
			\end{align}
			\begin{align}
			c_{24}=
				\begin{vmatrix}
					\text{ } \text{ }1 &\text{ } \text{ }3 &\text{ } \text{ }0 \text{ } \\
					-2 &-6 &\text{ } \text{ }3 \text{ } \\
					\text{ } \text{ }3 &\text{ } \text{ }5 &\text{ } \text{ }8 \text{ } \\
				\end{vmatrix}&=1(-6\times8-3\times5)-3(-2\times8-3\times3)+0 \notag \\
				&=-63+75+0=12 \notag
			\end{align}
			\begin{align}
			\hspace{1cm} c_{31}=
				\begin{vmatrix}
					\text{ } \text{ }3 &\text{ } \text{ }0 &-1 \text{ } \\
					\text{ } \text{ }1 &-2 &-1 \text{ } \\
					\text{ } \text{ }5 &\text{ } \text{ }8 &-3 \text{ } \\
				\end{vmatrix}&=3(-2\times(-3)-(-1)\times8)-0+(-1)(1\times8-(-2)\times5) \notag \\
				&=42+0-18=24 \notag
			\end{align}
			\begin{align}
			\hspace{1.2cm} c_{32}=-
				\begin{vmatrix}
					\text{ } \text{ }1 &\text{ } \text{ }0 &-1 \text{ } \\
					\text{ } \text{ }0 &-2 &-1 \text{ } \\
					\text{ } \text{ }3 &\text{ } \text{ }8 &-3 \text{ } \\
				\end{vmatrix}&=-\big[1(-2\times(-3)-(-1)\times8)-0+(-1)(0\times8-(-2)\times3)\big] \notag \\
				&=-\big[14+0-6\big]=-8 \notag
			\end{align}
			\begin{align}
			\hspace{0.4cm} c_{33}=
				\begin{vmatrix}
					\text{ } \text{ }1 &\text{ } \text{ }3 &-1 \text{ } \\
					\text{ } \text{ }0 &\text{ } \text{ }1 &-1 \text{ } \\
					\text{ } \text{ }3 &\text{ } \text{ }5 &-3 \text{ } \\
				\end{vmatrix}&=-0+1(1\times(-3)-(-1)\times3)-(-1)(1\times5-3\times3) \notag \\
				&=0+0-4=-4 \notag 
			\end{align}
			\begin{align}
			c_{34}=-
				\begin{vmatrix}
					\text{ } \text{ }1 &\text{ } \text{ }3 &\text{ } \text{ }0 \text{ } \\
					\text{ } \text{ }0 &\text{ } \text{ }1 &-2 \text{ } \\
					\text{ } \text{ }3 &\text{ } \text{ }5 &\text{ } \text{ }8 \text{ } \\
				\end{vmatrix}&=-\big[1(1\times8-(-2)\times5)-3(0\times8-(-2)\times3)+0\big] \notag \\
				&=-\big[18-18+0\big]=0 \notag
			\end{align}
			\begin{align}
			\hspace{1cm} c_{41}=-
				\begin{vmatrix}
					\text{ } \text{ }3 &\text{ } \text{ }0 &-1 \text{ } \\
					\text{ } \text{ }1 &-2 &-1 \text{ } \\
					-6 &\text{ } \text{ }3 &\text{ } \text{ }2 \text{ } \\
				\end{vmatrix}&=-\big[3(-2\times2-(-1)\times3)-0+(-1)(1\times3-(-2)\times(-6))\big] \notag \\
				&=-\left[-3+0+9\right]=-6 \notag
			\end{align}
			\begin{align}
			c_{42}=
				\begin{vmatrix}
					\text{ } \text{ }1 &\text{ } \text{ }0 &-1 \text{ } \\
					\text{ } \text{ }0 &-2 &-1 \text{ } \\
					-2 &\text{ } \text{ }3 &\text{ } \text{ }2 \text{ } \\
				\end{vmatrix}&=1(-2\times2-(-1)\times3)-0+(-1)(0\times3-(-2)\times(-2)) \notag \\
				&=-1+0+4=3 \notag
			\end{align}
			\begin{align}
			\hspace{0.8cm} c_{43}=-
				\begin{vmatrix}
					\text{ } \text{ }1 &\text{ } \text{ }3 &-1 \text{ } \\
					\text{ } \text{ }0 &\text{ } \text{ }1 &-1 \text{ } \\
					-2 &-6 &\text{ } \text{ }2 \text{ } \\
				\end{vmatrix}&=-\big[0+1(1\times2-(-1)\times(-2))-(-1)(1\times(-6)-3\times(-2))\big] \notag \\
				&=-\big[0+0+0\big]=0 \notag 
			\end{align}
			\begin{align}
			c_{44}=
				\begin{vmatrix}
					\text{ } \text{ }1 &\text{ } \text{ }3 &\text{ } \text{ }0 \text{ } \\
					\text{ } \text{ }0 &\text{ } \text{ }1 &-2 \text{ } \\
					-2 &-6 &\text{ } \text{ }3 \text{ } \\
				\end{vmatrix}&=1(1\times3-(-2)\times(-6))-3(0\times3-(-2)\times(-2))+0 \notag \\
				&=-9+12+0=3 \notag
			\end{align}
			Thus,
			\begin{align}
			\hspace{1.9cm} \text{adj}=
				\begin{pmatrix}
					\text{ } \text{ }54 &-25 &-8 &-9 \text{ } \\
					\text{ } \text{ }12 &\text{ } \text{ }0 &\text{ } \text{ }0 &\text{ } \text{ }12 \text{ } \\
					\text{ } \text{ }24 &-8 &-4 &\text{ } \text{ }0 \text{ } \\
					-6 &\text{ } \text{ }3 &\text{ } \text{ }0 &\text{ } \text{ }3 \text{ } 
				\end{pmatrix}^T=
				\begin{pmatrix}
					\text{ } \text{ }54 &\text{ } \text{ }12 &\text{ } \text{ }24 &-6 \text{ } \\
					-25 &\text{ } \text{ }0 &-8 &\text{ } \text{ }3 \text{ } \\
					-8 &\text{ } \text{ }0 &-4 &\text{ } \text{ }0 \text{ } \\
					-9 &\text{ } \text{ }12 &\text{ } \text{ }0 &\text{ } \text{ }3 \text{ } 
				\end{pmatrix} \notag
			\end{align}
			We also confirm the determinant calculated earlier by expanding on the second row of our 				original matrix and using the corresponding cofactors above.
			\smallskip
			\begin{align}
				\det &=-0\times c_{21}+1\times c_{22}-(-2)\times c_{23}+(-1)\times c_{24} \notag \\
				&=0\times12+1\times0+2\times0-1\times12 \notag \\
				&=0+0+0-12=-12 \notag
			\end{align}
			Hence,
			\begin{align}
				\hspace{1cm} 
				\begin{pmatrix}
					\text{ } \text{ }1 &\text{ } \text{ }3 &\text{ } \text{ }0 &-1 \text{ } \\
					\text{ } \text{ }0 &\text{ } \text{ }1 &-2 &-1 \text{ }  \\
					-2 &-6 &\text{ } \text{ }3 &\text{ } \text{ }2 \text{ } \\
					\text{ } \text{ }3 &\text{ } \text{ }5 &\text{ } \text{ }8 &-3 \text{ } 
				\end{pmatrix}^{-1}&=-\frac{1}{12}
				\begin{pmatrix}
					\text{ } \text{ }54 &\text{ } \text{ }12 &\text{ } \text{ }24 &-6 \text{ } \\
					-25 &\text{ } \text{ }0 &-8 &\text{ } \text{ }3 \text{ } \\
					-8 &\text{ } \text{ }0 &-4 &\text{ } \text{ }0 \text{ } \\
					-9 &\text{ } \text{ }12 &\text{ } \text{ }0 &\text{ } \text{ }3 \text{ } 
				\end{pmatrix} \notag \\
				&=
				\hspace{1.1cm}
				\begin{pmatrix}
					-\frac{9}{2} &-1 &-2 &\text{ } \text{ }\frac{1}{2} \text{ } \\
					\text{ } \text{ }\frac{25}{12} &\text{ } \text{ }0 &\text{ } \text{ }\frac{2}{3} &-\frac{1}{4} 
					\text{ } \\
					\text{ } \text{ }\frac{2}{3} &\text{ } \text{ }0 &\text{ } \text{ }\frac{1}{3} &\text{ } \text{ }0
					\text{ } \\
					\text{ } \text{ }\frac{3}{4} &-1&\text{ }\text{ }0 &-\frac{1}{4} \text{ } 
				\end{pmatrix} \notag
			\end{align}
		
		\end{enumerate}
	
	\item 

		\begin{enumerate}
		
			\item
			\begin{align}
				\hspace{1.3cm} \text{Let} \hspace{0.15cm}  A=
				\begin{pmatrix}
					2 &7 \\
					7 &2
				\end{pmatrix}, \hspace{0.2cm} 
				\text{then} \hspace{0.3cm} A-\lambda I=
				\begin{pmatrix}
					2 &7 \\
					7 &2
				\end{pmatrix}-
				\begin{pmatrix}
					\lambda &0 \\
					0 &\lambda
				\end{pmatrix}=
				\begin{pmatrix}
					2-\lambda &7 \\
					7 &2-\lambda
				\end{pmatrix} \notag
			\end{align} \\
			Since the characteristic equation for a square matrix $A$ is $\det(A-\lambda I)=0$, 
			\begin{align}
				(2-\lambda)^2-49=0 &\Rightarrow 2-\lambda=\pm7 \notag \\
				&\Rightarrow \lambda=-5,9 \notag
			\end{align}
			The associated eigenvectors are given below. \\
			
			$\lambda_1=-5:$
			\begin{align}
				\begin{pmatrix}
					2 &7 \\
					7 &2
				\end{pmatrix}
				\begin{pmatrix}
					x \\
					y
				\end{pmatrix}=-5
				\begin{pmatrix}
					x \\
					y
				\end{pmatrix} \notag
				\end{align}
				\begin{align}
				2x+7y=-5x \Rightarrow 7x=-7y &\Rightarrow x=-y \notag \\
				7x+2y=-5y \Rightarrow 7x=-7y &\Rightarrow x=-y \notag \\
				v_1= 
					\begin{pmatrix}
						-1 \text{ } \\
						\text{ } \text{ }1 \text{ }
					\end{pmatrix} \notag
			\end{align} 
			$\lambda_2=9:$
			\begin{align}
				\begin{pmatrix}
					2 &7 \\
					7 &2
				\end{pmatrix}
				\begin{pmatrix}
					x \\
					y
				\end{pmatrix}=9
				\begin{pmatrix}
					x \\
					y
				\end{pmatrix} \notag
				\end{align}
				\begin{align}
				2x+7y=9x \Rightarrow 7x=7y &\Rightarrow x=y  \notag \\
				7x+2y=9y \Rightarrow 7x=7y &\Rightarrow x=y  \notag \\
				v_2=
				\begin{pmatrix}
						\text{ }1 \text{ } \\
						\text{ }1 \text{ }
					\end{pmatrix} \notag
			\end{align} 
			
			\item 
			\begin{align}
				\hspace{2.16cm} \text{Let} \hspace{0.15cm}  A=
				\begin{pmatrix}
					3 &-2 \\
					1 &-1
				\end{pmatrix}, \hspace{0.2cm} 
				\text{then} \hspace{0.3cm} A-\lambda I=
				\begin{pmatrix}
					3 &-2 \\
					1 &-1
				\end{pmatrix}-
				\begin{pmatrix}
					\lambda &0 \\
					0 &\lambda
				\end{pmatrix}=
				\begin{pmatrix}
					3-\lambda &-2 \\
					1 &-1-\lambda
				\end{pmatrix} \notag
			\end{align}
			By the characteristic equation of a square matrix,
			\begin{align}
				(3-\lambda)(-1-\lambda)-(-2)\times1&=-3-3\lambda+\lambda+\lambda^2+2=0 \notag \\
				&\Rightarrow \lambda^2-2\lambda-1=0 \notag \\
				&\Rightarrow (\lambda-1)^2-2=0 \notag \\
				&\Rightarrow \lambda-1=\pm\sqrt{2} \notag \\
				&\Rightarrow \lambda=1\pm\sqrt{2} \notag
			\end{align}
			The associated eigenvectors are as follows. \\

			$\lambda_1=1-\sqrt{2}:$
			\begin{align}
				\begin{pmatrix}
					3 &-2\\
					1 &-1
				\end{pmatrix}
				\begin{pmatrix}
					x \\
					y
				\end{pmatrix}=(1-\sqrt{2})
				\begin{pmatrix}
					x \\
					y
				\end{pmatrix} \notag
				\end{align}
			\begin{align}
				3x-2y=x-\sqrt{2}x &\Rightarrow 2x+\sqrt{2}x=2y \notag \\
				&\Rightarrow x=\frac{2}{2+\sqrt{2}} \text{ }y \notag \\
				&\Rightarrow x=(2-\sqrt{2})y \notag \\
				x-y=y-\sqrt{2}y &\Rightarrow x=2y-\sqrt{2}y \notag \\
				&\Rightarrow x=(2-\sqrt{2})y \notag \\
				v_1&= 
					\begin{pmatrix}
						2-\sqrt{2} \\
						1 
					\end{pmatrix} \notag		
				\end{align} 
			$\lambda_2=1+\sqrt{2}:$
			\begin{align}
				\begin{pmatrix}
					3 &-2 \\
					1 &-1
				\end{pmatrix}
				\begin{pmatrix}
					x \\
					y
				\end{pmatrix}=(1+\sqrt{2})
				\begin{pmatrix}
					x \\
					y
				\end{pmatrix} \notag
				\end{align}
			\begin{align}
				3x-2y=x+\sqrt{2}x &\Rightarrow 2x-\sqrt{2}x=2y \notag \\
				&\Rightarrow x=\frac{2}{2-\sqrt{2}} \text{ }y \notag \\
				&\Rightarrow x=(2+\sqrt{2})y \notag \\
				x-y=y+\sqrt{2}y &\Rightarrow x=2y+\sqrt{2}y \notag \\
				&\Rightarrow x=(2+\sqrt{2})y \notag \\
				v_2&= 
					\begin{pmatrix}
						2+\sqrt{2} \\
						1
					\end{pmatrix} \notag	
			\end{align} \\
			
		\end{enumerate} 
		
	\item If $A$ is an invertible matrix, then $AA^{-1}=A^{-1}A=I$. To show that 
	$\text{adj}\hspace{0.05cm}A$ is also invertible, we therefore require
		
	$$\text{adj}\hspace{0.05cm}A \hspace{0.05cm} (\text{adj}\hspace{0.05cm}A)^{-1}=(\text{adj}
	\hspace{0.05cm}A)^{-1} \hspace{0.05cm} \text{adj}\hspace{0.05cm}A=I$$
	\\
	By definition $\text{adj} \hspace{0.05cm}A := \det (A)A^{-1}$.  It follows that 
	
		\begin{align}
			\det (A)A^{-1}\left(\det (A)A^{-1}\right)^{-1}&=\det (A)A^{-1} \det (A)^{-1}\left(A^{-1} \right)^{-1} 				\notag \\
			&=\det (A)A^{-1} \frac{1}{\det(A)}A \notag
		\end{align}
		\begin{align}
			\hspace{0.7cm} \text{since } \det(A)^{-1}&=\frac{1}{\det(A)} \text{ } \text{ as } \det(A) 
			\text{ is a scalar and} \notag \\
			\left(A^{-1}\right)^{-1}&=A \text{ } \text{ by the definition of inverse} \notag
		\end{align}
	Hence,
		\begin{align}
			\text{adj}\hspace{0.05cm}A \hspace{0.05cm} (\text{adj}\hspace{0.05cm}A)^{-1}=
			\frac{\det(A)A^{-1}A}{\det(A)} =A^{-1}A=I \notag
		\end{align}
	Similarly,
		\begin{align}
			(\text{adj} \hspace{0.05cm}A)^{-1} \hspace{0.05cm} \text{adj}\hspace{0.05cm}A&= \left(\det (A)			A^{-1}\right)^{-1} \det (A)A^{-1} \notag \\
			&=\det (A)^{-1}\left(A^{-1} \right)^{-1} \det (A)A^{-1} \notag \\
			&=\frac{A \hspace{0.05cm} \det (A)A^{-1}}{\det(A)}=AA^{-1}=I \notag
		\end{align}
	\\
	Thus we have shown that $\text{adj} \hspace{0.05cm}A$ is also an invertible matrix.
		
		\end{enumerate}

\end{document}
