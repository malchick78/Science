\documentclass[12pt]{amsart}

\usepackage{a4wide, amsxtra}

\usepackage[pdftex]{graphicx}

\usepackage{hyperref}

 \title{MATH101 Assignment 8}

 \author{Mark Villar}

\begin{document} 

\maketitle 

\begin{enumerate}
	
	\item 
	
		\begin{enumerate}
		
			\item  For $f: \mathbb{R} \longrightarrow \mathbb{R}, \hspace{0.3cm} x \longmapsto \sin x; 					\hspace{0.3cm} [0,2\pi]$
				\begin{enumerate}
					\item[(i)] $\text{ } \hspace{1cm} f'(x) = \cos x$
						\begin{align}
							\begin{cases}
								< 0 \hspace{0.3cm} \text{for} \hspace{0.3cm} (\frac{\pi}{2}, 
								\frac{3\pi}{2}) \notag \\
								= 0 \hspace{0.3cm} \text{for} \hspace{0.3cm} x=\frac{\pi}{2}, 
								\frac{3\pi}{2} \notag \\
								> 0 \hspace{0.3cm} \text{for} \hspace{0.3cm} (0, \frac{\pi}{2}) 
								\hspace{0.3cm} \text{and} \hspace{0.3cm} (\frac{3\pi}{2}, 2\pi)										\notag
							\end{cases}
						\end{align} 
						Thus $f$ is monotonically decreasing on $(\frac{\pi}{2},\frac{3\pi}{2})$ while
						it is monotonically increasing on $(0, \frac{\pi}{2})$ and $(\frac{3\pi}{2}, 2\pi)$. 
						It has critical points at $x=\frac{\pi}{2}, \frac{3\pi}{2}$. 
						
						As $f$ is an oscillating function, this pattern is repeated over $\mathbb{R}$
						with periodicity $2\pi$. \\
						
					\item[(ii)]
						$\text{ } \hspace{1.7cm} f''(x) = -\sin x$
						\begin{align}
							\begin{cases}
								< 0 \hspace{0.3cm} \text{for} \hspace{0.3cm} (0, \pi) \notag \\
								= 0 \hspace{0.3cm} \text{for} \hspace{0.3cm} x=0, \pi, 2\pi \notag \\
								> 0 \hspace{0.3cm} \text{for} \hspace{0.3cm} (\pi, 2\pi) \notag
							\end{cases}
						\end{align}
						Thus $f$ is concave down on $(0,\pi)$ while it is concave up on $(\pi,2\pi)$.
						It has points of inflection at $x=0,\pi,2\pi$. \\
				\end{enumerate}

			\item For $g: \mathbb{R} \longrightarrow \mathbb{R}, \hspace{0.3cm} x \longmapsto 
			e^x + e^{-x}$
				\begin{enumerate}
					\item[(i)] $\text{ } \hspace{2.35cm} g'(x) = e^x+(-1)e^{-x}$ \\
					 $\text{ } \hspace{3.25cm} = e^x-e^{-x}$
						\begin{align}
							\begin{cases}
								< 0 \hspace{0.3cm} \text{for} \hspace{0.3cm} x < 0 \notag \\
								= 0 \hspace{0.3cm} \text{for} \hspace{0.3cm} x = 0 \notag \\
								> 0 \hspace{0.3cm} \text{for} \hspace{0.3cm} x > 0 \notag
							\end{cases}
						\end{align} 
						Thus $f$ is monotonically decreasing when $x<0$ while it is monotonically 							increasing when $x>0$. It has a critical point at $x=0$. \\
											
					\item[(ii)] $\text{ } \hspace{2.3cm} g''(x) = e^x-(-1)e^{-x}$ \\
					$\text{ } \hspace{3.25cm} = e^x+e^{-x}$ \\ 
					$\text{ }\hspace{3.25cm} > 0 \hspace{0.3cm} \text{for} \hspace{0.3cm} x \in
					\mathbb{R}$.
						
					Thus $g$ is concave up for all $x$ and has a turning point at $x=0$. \\
					 
				\end{enumerate}

			\item For $h: \mathbb{R} \longrightarrow \mathbb{R}, \hspace{0.3cm} t \longmapsto 
			9t^3-9t^2+3$
			
				\begin{enumerate}
					\item[(i)] $\text{ } \hspace{0.96cm} h'(t) = 27t^2-18t$
						\begin{align}
							\begin{cases}
								< 0 \hspace{0.3cm} \text{for} \hspace{0.3cm} (0,\frac{2}{3}) \notag \\
								= 0 \hspace{0.3cm} \text{for} \hspace{0.3cm} t=0,\frac{2}{3} \notag \\
								> 0 \hspace{0.3cm} \text{for} \hspace{0.3cm} (-\infty,0) 
								\hspace{0.3cm} \text{and} \hspace{0.3cm} (\frac{2}{3}, \infty) \notag
							\end{cases}
						\end{align} 
						Thus $f$ is monotonically decreasing on $(0,\frac{2}{3})$ while it is 									monotonically increasing when $x<0$ and $x>\frac{2}{3}$. It has critical points 						at $t=0, \frac{2}{3}$. \\
										
					\item[(ii)] $\text{ } \hspace{2.1cm} h''(t) = 54t-18$
						\begin{align}
							\begin{cases}
								< 0 \hspace{0.3cm} \text{for} \hspace{0.3cm} (-\infty,\frac{1}{3}) 									\notag \\
								= 0 \hspace{0.3cm} \text{for} \hspace{0.3cm} t=\frac{1}{3} \notag \\
								> 0 \hspace{0.3cm} \text{for} \hspace{0.3cm} (\frac{1}{3},\infty) \notag
							\end{cases}
						\end{align} 
					Thus $h$ is concave down when $t<\frac{1}{3}$ while it is concave up when 
					$t>\frac{1}{3}$.	It has a point of inflection at $t=\frac{1}{3}$. \\
				
				\end{enumerate}

			\item For $k: \mathbb{R} \longrightarrow \mathbb{R}, \hspace{0.3cm} u \longmapsto 
			\ln(u^2+9)$
			
				\begin{enumerate}
					\item[(i)] $\text{ } \hspace{2.35cm} k'(u) = (\frac{1}{u^2+9})(2u) = \frac{2u}{u^2+9}$								 
						\begin{align}
							\begin{cases}
								< 0 \hspace{0.3cm} \text{for} \hspace{0.3cm} u<0 \notag \\
								= 0 \hspace{0.3cm} \text{for} \hspace{0.3cm} u=0 \notag \\
								> 0 \hspace{0.3cm} \text{for} \hspace{0.3cm} u>0 \notag
							\end{cases}
						\end{align} 
						Thus $f$ is monotonically decreasing when $u<0$ while it is monotonically 							increasing when $u>0$. It has a critical point at $u=0$. \\
										
					\item[(ii)] $\text{ } \hspace{0.65cm} k''(u) = \frac{(u^2+9)(2)-(2u)(2u)}{(u^2+9)^2} = 						\frac{18-2u^2}{(u^2+9)^2}$
																											\begin{align}
							\begin{cases}
								< 0 \hspace{0.3cm} \text{for} \hspace{0.3cm} (-\infty,-3) 
								\hspace{0.3cm} \text{and} \hspace{0.3cm} (3,\infty) \notag\notag \\
								= 0 \hspace{0.3cm} \text{for} \hspace{0.3cm} u=-3,3\notag \\
								> 0 \hspace{0.3cm} \text{for} \hspace{0.3cm} (-3,3) \notag
							\end{cases}
						\end{align} 
						
					Thus $k$ is concave down when $u<-3$ and $u>3$ while it is concave up when 
					$-3<u<3$.	 It has points of inflection at $u=-3,3$. \\
			
				\end{enumerate}

		\end{enumerate}
		
	\item For $f: (-1, \infty) \longrightarrow \mathbb{R}, \hspace{0.3cm} x \longmapsto \ln(1+x)-x$ 
		
		\begin{enumerate}
		
			\item $\text{ } \hspace{3.25cm} f'(x) = \frac{1}{1+x}-1= -\frac{x}{1+x}$
						\begin{align}
							\begin{cases}
								< 0 \hspace{0.3cm} \text{for} \hspace{0.3cm} (0,\infty) \notag \\
								= 0 \hspace{0.3cm} \text{for} \hspace{0.3cm} x = 0 \notag \\
								> 0 \hspace{0.3cm} \text{for} \hspace{0.3cm} (-1, 0) \notag
							\end{cases}
						\end{align} 
			
			Thus $f$ is monotonically decreasing on $\mathbb{R}_0^+$ since $f'(x)<0$ for all $x>0$. As 				$f$ is decreasing, %its largest value will occur at its critical point $x=0$, whereby
			$$f(x) \le f(0) = 0$$
			Hence, for all $x\ge0:$
				\begin{align}
					\ln(1+x)-x \le 0 \notag \\
					\ln(1+x) \le x \notag
				\end{align}
			
			\item Let $g: (-1, \infty) \longrightarrow \mathbb{R}, \hspace{0.3cm} x \longmapsto 
			\ln(1+x)-x+\frac{x^2}{2}$ \\
			$\text{ } \hspace{2.05cm} g'(x) = -\frac{x}{1+x}+x=\frac{x^2}{1+x}$
				\begin{align}
					 \begin{cases}
						= 0 \hspace{0.3cm} \text{for} \hspace{0.3cm} x=0 \notag \\
						> 0 \hspace{0.3cm} \text{for} \hspace{0.3cm} (-1,0)  \hspace{0.3cm} \text{and} 							\hspace{0.3cm} (0,\infty) \notag
					\end{cases}
				\end{align} 
		
			Thus $g$ is monotonically increasing on $\mathbb{R}_0^+$ since $g'(x)>0$ for all $x>0$.  As 
			$g$ is increasing, %its smallest value will occur at its critical point $x=0$, whereby
			$$g(x) \ge g(0) = 0$$
			Hence, for all $x\ge0:$
				\begin{align}
					\ln(1+x)-x+\frac{x^2}{2} \ge 0 \notag \\
					\ln(1+x) \ge x-\frac{x^2}{2} \notag
				\end{align}
				
			Let $h: (-1, \infty) \longrightarrow \mathbb{R}, \hspace{0.3cm} x \longmapsto 
			\ln(1+x)-x+\frac{x^2}{2}-\frac{x^3}{3}$ \\
			$\text{ } \hspace{3.3cm} g'(x) = \frac{x^2}{1+x}-x^2=-\frac{x^3}{1+x}$
				\begin{align}
					 \begin{cases}
						< 0 \hspace{0.3cm} \text{for} \hspace{0.3cm} (0, \infty) \\
						= 0 \hspace{0.3cm} \text{for} \hspace{0.3cm} x=0 \notag \\
						> 0 \hspace{0.3cm} \text{for} \hspace{0.3cm} (-1,0) \notag 
					\end{cases}
				\end{align} 
		
			Thus $h$ is monotonically decreasing on $\mathbb{R}_0^+$ since $h'(x)<0$ for all $x>0$.  As 
			$h$ is decreasing, %its largest value will occur at its critical point $x=0$, whereby
			$$h(x) \le h(0) = 0$$
			Hence, for all $x\ge0:$
				\begin{align}
					\ln(1+x)-x+\frac{x^2}{2}-\frac{x^3}{3} \le 0 \notag \\
					\ln(1+x) \le x-\frac{x^2}{2}+\frac{x^3}{3} \notag
				\end{align}	
				
			Thus we have shown that 
			$$x-\frac{x^2}{2} \le \ln(1+x) \le x-\frac{x^2}{2}+\frac{x^3}{3}$$ \\
			 
		\end{enumerate}
								
	\item 
	
		\begin{enumerate}
		
			\item For $f: \mathbb{R} \longrightarrow \mathbb{R}, \hspace{0.3cm} x \longmapsto 							3x^5-20x^3+45x+1$ \\
			
				The domain of $f$ is $\mathbb{R}$ which contains no boundary points.  Moreover, since					$f$ is a polynomial function it is differentiable everywhere. Hence, \\
				
				$f'(x) = 15x^4-60x^2+45=15(x^2-3)(x-1)(x+1)=0$ \\
				$\text{ } \hspace{5cm} \text{if and only if} \hspace{0.3cm} x \in \{\pm1,\pm \sqrt{3}\}$ \\
				
				$f''(x) = 60x^3-120x$ \\ %=60x(x^2-2)$ \\	
				
				Then $f''(-\sqrt{3})=-60\sqrt{3}, \hspace{0.3cm} f''(-1)=60, \hspace{0.3cm} f''(1)=-60, 
				\hspace{0.3cm} f''(\sqrt{3})=60\sqrt{3}$ \\
				
				Thus $f$ has relative minima at $x=-1,\sqrt{3}$ and relative maxima at $x=-\sqrt{3},1$. \\
				
				$f(-\sqrt{3})=-12\sqrt{3}+1 \approx -19.78, \hspace{0.3cm} f(-1)=-27, \hspace{0.3cm} 
				f(1)=29,$ \\
				$f(\sqrt{3})=12\sqrt{3}+1 \approx 21.78$ \\
															
			\item For $g: \mathbb{R} \longrightarrow \mathbb{R}, \hspace{0.3cm} t\longmapsto t^2e^{2t}$\\
			
				The domain of $g$ is $\mathbb{R}$ which contains no boundary points. Additionally, 
				$g$ is differentiable everywhere. Hence,\\
				
				$g'(t) = 2te^{2t}+2t^2e^{2t} = 2te^{2t}(1+t)=0$ \\
				$\text{ } \hspace{4.1cm} \text{if and only if} \hspace{0.3cm} t \in \{-1,0\}$ \\
				
				$g''(t) = 4te^{2t}+2e^{2t}+4t^2e^{2t}+4te^{2t} = 4t^2e^{2t}+8te^{2t}+2e^{2t}$ \\
				%=2e^{2t}(2t^2+4t+1)$
				
				Then $g''(-1)=-\frac{2}{e^2}, \hspace{0.3cm} g''(0)=2 $ \\
				
				Thus $g$ has relative maximum at $x=-1$ and relative minimum at $x=0$. \\
				
				$g(-1) = \frac{1}{e^2} \approx 0.135, \hspace{0.3cm} g(0) = 0$ \\
				
			\item For $h: (-3,\infty) \longrightarrow \mathbb{R}, \hspace{0.3cm} x \longmapsto 							\sqrt{x^3+3x^2+3}$ \\
			
				We observe that $h$ is differentiable over its domain $(-3,\infty)$. Hence,\\
				
				$h'(x) = \left(\frac{1}{2}\right)(x^3+3x^2+3)^{-\frac{1}{2}}(3x^2+6x) =
				\frac{3x^2+6x}{2\sqrt{x^3+3x^2+3}} = \frac{3x(x+2)}{2\sqrt{x^3+3x^2+3}} = 0$
				\\
				\\
				$\text{ } \hspace{7.5cm} \text{if and only if} \hspace{0.3cm} x \in \{-2,0\}$ \\
				
				(The only critical points are given by $3x^2(x+2)=0$ since $x^3+3x^2+3>0)$ \\
				
				$h''(x) = \frac{1}{2}\left[\left(-\frac{1}{2}\right)(3x^2+6x)(x^3+3x^2+3)^{-\frac{3}{2}}
				(3x^2+6x)+(x^3+3x^2+3)^{-\frac{1}{2}}(6x+6)\right]$ \\
				
				$\text{ } \hspace{0.845cm} = \frac{6x+6}{2\sqrt{x^3+3x^2+3}}-
				\frac{(3x^2+6x)^2}{4(x^3+3x^2+3)^\frac{3}{2}}$ \\
				
				Then $h''(-2) = -\frac{3\sqrt{7}}{7}, \hspace{0.3cm} h''(0)=\sqrt{3} $ \\
				
				Thus $h$ has relative maximum at $x=-2$ and relative minimum at $x=0$. \\
				
				$h(-2)=\sqrt{7}, \hspace{0.3cm} h(0) = \sqrt{3}$ \\
				
		\end{enumerate}
	
	\item For $n \in \mathbb{N}$ and $x\ge0$, let $P(n)$ be the proposition: 
		$$\ln(1+x) \ge x-\frac{x^2}{2}+\frac{x^3}{3}-\frac{x^4}{4}+...-\frac{x^{2n}}{2n}$$
		Equivalently,
		$$\ln(1+x) \ge \sum_{n=1}^\infty \Bigg[\frac{x^{2n-1}}{2n-1} - \frac{x^{2n}}{2n}\Bigg]$$
		$\mathbf{n=1:}$
			\begin{align}
				\frac{x^{2n-1}}{2n-1} - \frac{x^{2n}}{2n}= \frac{x^{2-1}}{2-1}-\frac{x^2}{2} = x - \frac{x^2}{2} 				\notag
			\end{align} \\
		We showed in Question 2(b) that $\ln(1+x) \ge x-\frac{x^2}{2}$. Therefore $P(1)$ is true. 
		\\
		$\mathbf{n \ge 2:}$ We make the inductive hypothesis that $P(n)$ is true for $n=k$.
		$$\ln(1+x) \ge \sum_{k=1}^\infty \Bigg[\frac{x^{2k-1}}{2k-1} - \frac{x^{2k}}{2k}\Bigg]$$
		We now prove $P(k+1):$ 
		$$\text{Let} \hspace{0.2cm} j(x)=\ln(1+x)-x+\frac{x^2}{2}-\frac{x^3}{3}+\frac{x^4}{4}-...
		+\frac{x^{2k}}{2k}-\frac{x^{2k+1}}{2k+1}$$
		Then $j$ is differentiable over its domain $(-1,\infty)$ and
			\begin{align}
				j'(x)&=\frac{1}{1+x}-1+x-x^2+x^3-...+x^{2k-1}-x^{2k} \notag \\
				& = -\frac{x}{1+x}+\sum_{k=1}^\infty \bigg[x^{2k-1}-x^{2k}\bigg] = -\frac{x}{1+x} +
				\frac{1}{x^2+x} = \frac{1-x}{x} \notag
			\end{align} \\
		It follows that 
			\begin{align}
				j'(x)= \notag
				\begin{cases}
					<0 \hspace{0.3cm} \text{for} \hspace{0.3cm} x>1 \notag \\
					=0 \hspace{0.3cm} \text{for} \hspace{0.3cm} x=1\notag \\
					>0 \hspace{0.3cm} \text{for} \hspace{0.3cm} 0<x<1 \notag 
				\end{cases}
			\end{align}
		\\		
		Thus $j$ is monotonically increasing for $0<x<1$ and as such,
		$$j(x) \ge j(0)=0$$
		Hence, for $0<x<1:$
		$$\text{ } \hspace{0.5cm} \ln(1+x)-x+\frac{x^2}{2}-\frac{x^3}{3}+\frac{x^4}{4}-...+\frac{x^{2k}}{2k}-
		\frac{x^{2k+1}}{2k+1} \ge 0$$ 
		
		$$\ln(1+x) \ge x-\frac{x^2}{2}+\frac{x^3}{3}-\frac{x^4}{4}+...-\frac{x^{2n}}{2n}+
		\frac{x^{2k+1}}{2k+1}$$
		\\
		Therefore $P(k+1)$ is true whenever $P(k)$ is true. So by the principle of mathematical induction, 			for all $n \in \mathbb{N}:$ \\
		$$x-\frac{x^2}{2}+\frac{x^3}{3}-\frac{x^4}{4}+...-\frac{x^{2n}}{2n} \le \ln(1+x)$$
		\\
		We then let $S(n)$ be the statement:
		$$\ln(1+x) \le x-\frac{x^2}{2}+\frac{x^3}{3}-\frac{x^4}{4}+...+\frac{x^{2n-1}}{2n-1}$$
		Equivalently,
		$$\ln(1+x) \le x-\sum_{n=1}^\infty \Bigg[\frac{x^{2n}}{2n} - \frac{x^{2n+1}}{2n+1}\Bigg]$$
		$\mathbf{n=1:}$
			\begin{align}
				x- \frac{x^{2n}}{2n} + \frac{x^{2n+1}}{2n+1} = x-\frac{x^2}{2}+\frac{x^3}{3} \notag
			\end{align} \\
		We showed in Question 2(b) that $\ln(1+x) \le x-\frac{x^2}{2}+\frac{x^3}{3}$. Therefore $S(1)$ is 			true. 
		\\
		$\mathbf{n \ge 2:}$ We make the inductive hypothesis that $S(n)$ is true for $n=k$.
		$$\ln(1+x) \le x-\sum_{k=1}^\infty \Bigg[\frac{x^{2k}}{2k} - \frac{x^{2k+1}}{2k+1}\Bigg]$$
		We now prove $S(k+1):$ 
		 $$\text{Let} \hspace{0.2cm} l(x)=\ln(1+x)-x+\frac{x^2}{2}-\frac{x^3}{3}+\frac{x^4}{4}-...
		-\frac{x^{2k-1}}{2k-1}+\frac{x^{2k}}{2k}$$
		Then $l$ is differentiable over its domain $(-1,\infty)$ and
			\begin{align}
				l'(x)&=\frac{1}{1+x}-1+x-x^2+x^3-...+x^{2k-2}-x^{2k-1} \notag \\
				& = -\frac{x}{1+x}+\sum_{k=1}^\infty \bigg[x^{2k-2}-x^{2k-1}\bigg] = -\frac{x}{1+x} +
				\frac{1}{x^3+x^2} = \frac{1-x^3}{x^3+x^2} \notag
			\end{align} \\
		It follows that 
		\begin{align}
				l'(x)= \notag
				\begin{cases}
					<0 \hspace{0.3cm} \text{for} \hspace{0.3cm} x>1 \notag \\
					=0 \hspace{0.3cm} \text{for} \hspace{0.3cm} x=1\notag \\
					>0 \hspace{0.3cm} \text{for} \hspace{0.3cm} 0<x<1 \notag 
				\end{cases}
			\end{align}
		\\				
		Thus $l$ is monotonically decreasing for $x>1$ and as such,
		$$l(x) \le l(0)=0$$
		Hence, for $x>1:$
		$$\text{ } \hspace{0.5cm} \ln(1+x)-x+\frac{x^2}{2}-\frac{x^3}{3}+\frac{x^4}{4}-...-\frac{x^{2k-1}}{2k-1} +		\frac{x^{2k}}{2k}\le 0$$ 
		
		$$\ln(1+x) \le x-\frac{x^2}{2}+\frac{x^3}{3}-\frac{x^4}{4}+...+\frac{x^{2k-1}}{2k-1}-\frac{x^{2k}}{2k}$$
		\\
		Therefore $S(k+1)$ is true whenever $S(k)$ is true. So by the principle of mathematical induction, 			for all $n \in \mathbb{N}:$ \\
		$$\ln(1+x) \le x-\frac{x^2}{2}+\frac{x^3}{3}-\frac{x^4}{4}+...+\frac{x^{2n-1}}{2n-1}$$ \\
		Finally, we have shown that
		$$x-\frac{x^2}{2}+\frac{x^3}{3}-...-\frac{x^{2n}}{2n} \le \ln(1+x) \le x-\frac{x^2}{2}+\frac{x^3}{3}-...+
		\frac{x^{2n-1}}{2n-1}$$
		
\end{enumerate}
	
\end{document}
	